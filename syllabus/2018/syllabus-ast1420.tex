\documentclass{article}

\usepackage{float}
\usepackage{hyperref}
\usepackage{url}
\usepackage{amsmath,amssymb}

\include{vc}

\pagestyle{empty}

\baselineskip 18pt
\textwidth 6.25in
\textheight 8.5in
\oddsidemargin 0.1in
\evensidemargin 0.1in
\marginparwidth 0in
\marginparsep 0in
\topmargin -.5in
\topskip -.5in
\parindent 0in
\parskip 1pt

\begin{document}

\begin{center}
  \LARGE{\scshape{AST1420: Galactic structure and dynamics}}\\[5pt]
  \Large{\scshape{Fall 2018}}\\[5pt]
  \large{(last updated: \today; rev. \githash)}\\[25pt]
\end{center}

\section*{Course description}

This graduate-level course provides an introduction to galaxies and
their properties. The focus of the course is on the physical
understanding of the fundamental processes that shape galaxies and
their constituents and much attention will go to various
manifestations of the gravitational force, arguably the most important
force shaping galaxies. We will also focus on learning the basic
theoretical tools and observational data sets used in the study of
galaxies.

\section*{Logistics}

\begin{itemize}

  \item {\bf Meeting time / place:} Tue/Fri, 2:10--3:30pm, Room: AB
    113.\\ Weeks of Oct 9 and 16, we will meet for 2 hours on Tue
    (2--4pm), but not on Fri.

  \item {\bf Instructor:} Jo Bovy, AB 229.

  \item {\bf Email:} \href{mailto:jo.bovy@utoronto.ca}{jo.bovy@utoronto.ca}

  \item {\bf Office hours:} Stop by my office or by appointment.

  \item {\bf Course website:} \url{https://github.com/jobovy/AST1420}.

\end{itemize}

\section*{Learning objectives}

After this course you should understand

\begin{itemize}

  \item the different constituents of galaxies, their typical
    properties (density, morphology, kinematics, etc.), and their
    relation to one another.

  \item the basic dynamical properties of mass distributions:
    dynamical time, relaxation time, circular velocity, escape speed,
    and the important differences between spherical, axisymmetric, and
    triaxial mass distributions.

  \item the basic properties of orbits for spherical, axisymmetric,
    and non-axisymmetric mass distributions; the importance of
    conservation laws and integrals of the motion in characterizing
    orbits and galaxies.

  \item the properties of close-to-circular orbits (epicycle
    approximation) and the dynamics of the solar neighborhood.

  \item rotation curves and the distribution of dark matter in
    galaxies.

  \item equilibrium states of spherical and axisymmetric stellar
    systems and how to use these to observationally measure the mass and
    orbital distributions of galaxies.

  \item the formation and structure of dark matter halos in the
    $\Lambda$CDM paradigm.

  \item $N$-body modeling.

  \item basic galactic chemical evolution.

  \item how and why bars and spiral structure forms; how galaxies form
    and grow through violent relaxation, phase-mixing, gas accretion,
    mergers, and dynamical friction.

  \item how we think disk and elliptical galaxies form.

\end{itemize}

\section*{Reading}

A set of lecture notes will be posted on the course website throughout the semester. For additional reading, we will mostly be using

\begin{itemize}

  \item Binney \& Tremaine, \emph{Galactic Dynamics, 2nd Edition},
    2008, Princeton University Press. Errata can be found
    \href{https://www-thphys.physics.ox.ac.uk/people/JamesBinney/web/index\_files/BT2errors.pdf}{here}.

  \item Binney \& Merrifield, \emph{Galactic Astronomy},
    1998, Princeton University Press. Errata can be found
    \href{http://www-thphys.physics.ox.ac.uk/people/JamesBinney/bmerrors.pdf}{here}.

  \item Mo, van den Bosch, \& White, \emph{Galaxy Formation and Evolution},
    2010, Cambridge University Press. Errata can be found
    \href{http://people.umass.edu/hjmo/book/errata.pdf}{here}.

\end{itemize}

\section*{Grading scheme}

\begin{itemize}

  \item {\bf Assignments:} 30\,\% over three assignments; see course website for due dates.

  \item {\bf Participation:} 20\,\%

  \item {\bf Presentations:} 20\,\%

  \item {\bf Take-home final + oral exam:} 30\,\%

You are allowed to (and are encouraged to!) work together with
classmates on the assignments, but each student must hand in an
independent write-up of their solutions. The take-home final should be
your own work. Solutions must be written up in a detailed enough
manner to demonstrate that you understand each step. The oral exam
will consist of a discussion of the take-home final and questions from
the General Qualifying Exam.

\end{itemize}

\section*{Academic integrity}

From Appendix D of the Academic Integrity Handbook:
\begin{quote}
  Academic integrity is one of the cornerstones of the University of
  Toronto. It is critically important both to maintain our community
  which honours the values of honesty, trust, respect, fairness, and
  responsibility and to protect you, the students within this
  community, and the value of the degree towards which you are all
  working so diligently.  

  According to Section B of the University of
  Toronto's Code of Behaviour on Academic Matter
  (\url{http://www.governingcouncil.utoronto.ca/policies/behaveac.htm})
  which all students are expected to read and by which they are
  expected to abide, it is an offence for students to:
  \begin{itemize}
    \item Use someone else's ideas or words in their own work without
      acknowledging explicitly that those ideas/words are not their
      own with a citation and quotation marks, i.e. to commit
      plagiarism.
  \item Include false, misleading, or concocted citations in their
    work.
  \item Obtain unauthorized assistance on any assignment. 
  \item Provide unauthorized assistance to another students. This
    includes showing another student your own work.
  \item Submit their own work for credit in more than one course
      without the permission of the instructors.
  \end{itemize}

  There are other offenses covered under the Code, but these are the
  most common. You are instructed to respect these rules and the
  values that they protect.
\end{quote}

\section*{Schedule}

\begin{itemize}

  \item {\bf Week 1:} Introduction to galactic structure; overview of
    background knowledge.

  \item {\bf Week 2:} General properties of gravitational potentials;
    properties and examples of spherical mass distributions; basics of
    classical mechanics; orbits in spherical potentials.

  \item {\bf Week 3:} Galaxies as collisionless sytems; equilibrium
    configurations of spherical systems; virial theorem; collisionless
    Boltzmann equation; spherical Jeans equations; spherical
    distribution functions; applications: masses of spherical systems.

  \item {\bf Week 4:} Properties of disky mass distributions; orbits
    in axisymmetric potentials; surfaces of section; equilibrium
    configurations of disks.

  \item {\bf Week 5:} Rotation curves; gas kinematics in the Milky
    Way; the Milky Way's rotation curve; Oort constants; the dynamics
    of the Solar neighborhood.

  \item {\bf Week 6:} Formation and evolution of dark matter halos;
    violent relaxation; phase-mixing.

  \item {\bf Week 7:} Spheroidal and triaxial mass distributions;
    orbits in these mass distributions; Schwarzschild modeling; orbits
    in these mass distributions; the internal structure of elliptical
    galaxies.

  \item {\bf Week 8:} chemical evolution of galaxies; age--abundance
    relations in the solar neighborhood; stellar population synthesis.

  \item {\bf Week 9:} Numerical methods; $N$-body modeling.

  \item {\bf Week 10:} Student presentations.

  \item {\bf Week 11:} Dynamical instabilities; bars; spiral arms;
    mergers and dynamical friction; tides.

\end{itemize}

\section*{Is there class this week? What's due?}

\begin{table}[H]
\begin{center}
\begin{tabular}{lcccc}
Week & Dates &  \quad Tue? \quad & \quad Fri? \quad & \quad Due on Tue? Other notes\\
\hline
\hline
1 & Sep 11 / Sep 14 & Yes & Yes & \\ % Intro
2 & Sep 18 / Sep 21 & Yes & Yes & \\ % Gravity / spherical orbits
3 & Sep 25 / Sep 28 & Yes & Yes & \\ % Equilibrium, spherical mass
4 & Oct 02 / Oct 05 & Yes & Yes & \\ % Disks, disk mass, disk orbits
5 & Oct 09 / Oct 12 & Yes & {\bf No} & Assignment 1 / 2-hr Tue lecture \\ % galactic rotation
6 & Oct 16 / Oct 19 & Yes & {\bf No} & Presentation topic  / 2-hr Tue lecture \\ % DM halos
7 & Oct 23 / Oct 26 & Yes & {\bf No} & Assignment 2 \\ % Spheroidal and triaxial, ellip.
 & Oct 30 / Nov 02 & {\bf No} & {\bf No} & \\ 
8 & Nov 06 / Nov 09 & Yes & Yes & \\ % chem. ev.
9 & Nov 13 / Nov 16 & Yes & Yes &  Assignment 3 \\ % N-body
10 & Nov 20 / Nov 23 & Yes & Yes & Presentations this week\\ 
11 & Nov 27 / Nov 30 & Yes & Yes % Bars, spiral, df, tides
\end{tabular}
\end{center}
\end{table}

\end{document}

