\documentclass[12pt]{article}
\usepackage[letterpaper,margin=1in]{geometry}
\usepackage{hyperref}
\usepackage{amsmath}
\begin{document}
\begin{center}
{\bf \LARGE AST1420 ``Galactic Structure and Dynamics'' Problem Set 1}\\[7pt]
\emph{Due on Oct. 9 at the start of class}\\[7pt]
\end{center}

Most of the exercises in this problem set must be solved on a computer
and the best way to hand in the problem set is as an \texttt{ipython
  notebook}. Rather than sending me the notebook, you can upload it to
\texttt{GitHub}, which will automatically render the notebook. Rather
than starting a repository for a single notebook, you can upload your
notebook as a \texttt{\href{https://gist.github.com/}{gist}}, which
are version-controlled snippets of code that can optionally be made
private.

If you want to upload your notebook as a gist from the command-line,
you can use the package \href{http://github.com/defunkt/gist}{at this
  http URL} and use it as follows. Log into your \texttt{GitHub}
account:\\

\texttt{gist --login}\\

and then upload your notebook
\texttt{AST1420\_2018\_PS1\_YOURNAME.ipynb} as\\

\texttt{gist -p AST1420\_2018\_PS1\_YOURNAME.ipynb}\\

(the \texttt{-p} option will make the gist private). If you want to
make further changes, you can clone your gist in a separate directory
and use it as you would any other git repository. \emph{Please re-run
  the entire notebook (with \texttt{Cell > Run All}) after re-starting
  the notebook kernel before uploading it}; this will make sure that
the input and output are fully consistent.

If you are unfamiliar with notebooks, you can also hand in a
traditional write-up, but you also need to send in well-commented code
for how you solved the problems. Thus, notebooks are strongly preferred :-)\\

\noindent{\bf Problem 1:} The Local Group Timing Argument. This
argument about the total mass of the Local Group by Kahn \& Woltjer
was very influential in early discussions of dark matter. The basic
argument is given in the lecture notes, where we also performed a
simple estimate of the mass of the Local Group implied by the current
separation and relative velocity of Andromeda with respect to the
Milky Way and the age of the Universe. We'll do a better job here.\\

(a) As discussed in the notes, we can model the system as that of the
displacement vector with respect to the center of mass in a point-mass
potential with mass equal to the sum of the Milky Way and Andromeda
masses. Solve the full Keplerian equation of motion for a radial orbit
(eccentricity equal to 1) for a present-day separation of 740 kpc,
relative velocity $-125\,\mathrm{km\,s}^{-1}$, and age of the Universe
13.7 Gyr. What is the exact mass of the Local Group implied by these
values? What is the maximum separation between the Milky Way and
Andromeda in the past? (Hint: use the parametric solution in terms of
the eccentric anomaly; see BT08.)\\

(b) The modeling in the notes and in part (a) assumes a matter-only
Universe. However, our Universe contains both matter and dark energy
(and other components, but they can be neglected for the purpose of
this problem). The contribution of dark energy to the gravitational
acceleration is $\Lambda r / 3$, where $\Lambda = 8\pi
G\rho_\Lambda/c^2$, with $\rho_\Lambda$ the cosmological dark
energy density. By using the expression for the radial period of an
orbit in a spherical potential, a similar expression for the time
between apocenter and the present radius, and energy conservation,
numerically solve for the mass of the Local Group. In this case, what
is the maximum separation? Assume that $\Omega_\Lambda = 0.7$ and $H_0
= 70\,\mathrm{km\,s}^{-1}\,\mathrm{Mpc}^{-1}$.\\

\noindent{\bf Problem 2:} (Inspired by BT08 problem 3.6) Let's
consider what happens when we add energy to an orbit in a spherical
potential.\\

\noindent{\bf (a)} Consider an orbit in a spherical isochrone
potential with $b=1$ (pick an orbit that explores $r \approx 1$ and is
not too close to circular). Using orbit integration in \texttt{galpy},
add an instantaneous velocity offset when the orbit is at its
pericenter radius. Investigate what happens to the pericenter radius
of the resulting orbit. Is it larger or smaller than the original
pericenter radius? It is useful to consider the special cases where
(i) you only change the radial velocity and (ii) you only change the
tangential velocity (or equivalently the angular momentum). Does the
answer change if you consider different orbits?\\

\noindent{\bf (b)} Argue why the behavior you saw in part (a) is true
for \emph{any} orbit in \emph{any} spherical potential. (Hint:
consider the special cases and what happens to the effective
potential). You can illustrate your argument with the orbit(s) that
you investigated in (a), but make it clear why the behavior is general.\\

\noindent{\bf (c)} What happens to the apocenter radius and the
eccentricity in (a)? Investigate numerically and explain what is
happening.\\

\noindent{\bf (d)} Does the behavior you see change if you apply the
velocity offset at different parts of the orbit (e.g., at apocenter,
or at a random point along the orbit)?\\

\noindent{\bf Problem 3:} The velocity structure of dark matter
halos. We have discussed the spherical Jeans equation and its
dependence on the anisotropy parameter $\beta$. Let's investigate its
effect on velocities in dark matter halos.\\

\noindent{\bf (a)} The simplest model for a dark matter halo is the
\emph{logarithmic} model: $\Phi(r) = v^2_c\,\ln r$. What is the radial
velocity dispersion for a constant $\beta$?\\

\noindent{\bf (b)} A more realistic model for the anisotropy accounts
for the fact that orbits are more isotropic in the inner regions, where
dark matter has been orbiting for many dynamical times, and more
radial in the outer regions, where dark matter has only just been
accreted onto the halo. In the Osipkov--Merritt model, the anisotropy depends on radius as

\begin{equation}\label{eq:beta-osipkov-merritt}
  \beta(r) = \frac{r^2}{r^2+r_a^2}\,,
\end{equation}

where $r_a$ is parameter that sets the transition between isotropic
and radial orbits. Setting $r_a = 1$, compute the radial velocity
dispersion for the logarithmic model with $v_c = 1$ (perhaps
numerically). By comparing to the results in (a), does your solution
make sense?\\

\noindent{\bf (c)} As discussed in class, an often preferred model for
the dark matter density profile is the NFW profile. Setting the
parameters of the NFW profile such that the circular velocity is equal
to one at $a/2$, where $a$ is the scale radius, compute the radial
velocity dispersion profile for $\beta = 0$, $\beta = 0.5$, $\beta =
1$, and the form of Equation \eqref{eq:beta-osipkov-merritt} with $r_a
= 40a/3$. Plot your solution on a logarithmic grid from $r = a/10$ to
$r=100a$, where $a$ is the scale radius.

\end{document}
