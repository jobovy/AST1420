\documentclass[12pt]{article}
\usepackage[letterpaper,margin=1in]{geometry}
\usepackage{hyperref}
\usepackage{amsmath,amssymb}
\begin{document}
\begin{center}
{\bf \LARGE AST1420 ``Galactic Structure and Dynamics'' Problem Set 1}\\[7pt]
\emph{Due on Feb. 7 by 5pm}\\[7pt]
\end{center}

Some of the exercises in this problem set must be solved on a computer
and a good way to hand in the problem set is as a \texttt{jupyter
  notebook}. \emph{Please re-run the entire notebook (with \texttt{Cell
    > Run All}) after re-starting the notebook kernel before sending
  it in}; this will make sure that the input and output are fully
consistent. You can also send in a traditional write-up in LaTeX as a PDF, 
but then you also need to send in well-commented code for how you solved 
the numerical problems. Thus, notebooks are preferred :-)\\

\noindent{\bf Problem 1:} We introduced the virial radius and virial mass 
in the context of the NFW profile, but we can similarly define the virial 
quantities for any radial mass profile. A simple model for galaxies and 
their dark-matter halos is the logarithmic potential from 
Equation (3.58). Again defining the virial 
radius as the radius within which the average density is \(\Delta_v\) 
times the critical density, demonstrate that the following useful relation 
holds
\begin{equation*}
\begin{split}v_{200} = 10H\,r_{200}\,,\end{split}
\end{equation*}
where \(r_{200}\) is the virial radius assuming \(\Delta_v = 200\), \(v_{200}\) is 
the circular velocity at the virial radius, and \(H\) is the Hubble constant. 
Identifying \(v_{200}\) with the asymptotic value of the flat rotation curve for 
other profiles, this relation gives a quick way to estimate the virial radius of a 
galaxy.\\

\noindent{\bf Problem 2:} A model for dark matter is that it is a light scalar 
particle and, in this case, it is possible that dark matter within galaxies settles 
into a Bose--Einstein condensate. Under a certain set of assumptions 
(B\"{o}hmer \& Harko 2007), such a dark-matter model is equivalent to a fluid with an 
equation of state \(p = (2\pi \hbar^2 a/m^3)\,\rho^2\) between pressure 
\(p\) and density \(\rho\) that is in hydrostatic equilibrium 
\(\mathrm{d} p / \mathrm{d} r = -\rho\,\mathrm{d} \Phi / \mathrm{d} r\). The 
fundamental parameters of this theory are the particle mass \(m\) and the scattering 
length \(a\). Considering this dark-matter model in isolation 
(i.e., ignoring any baryons), demonstrate that the dark-matter density profile is 
\(\rho(r) \propto \sin(kr)/(kr)\) and determine \(k\) in terms of \(m\), \(a\), and 
universal constants. The Bose-Einstein condensate has a boundary where 
its density goes to zero; determine this boundary—the radius of the dark-matter halo. 
Finally, compute and plot the rotation curve of this model for parameters chosen such 
that the radius of the halo is 15 kpc and its mass is \(10^{11}\,M_\odot\).\\

\noindent{\bf Problem 3:} The precession of the perihelion of Mercury. A famous 
postdiction by the general theory of relativity (GR) is that the perihelion of 
Mercury precesses more than expected using Newtonian gravity (Mercury's perihelion 
precesses through the Newtonian influence from the other planets). Specifically, 
the azimuthal angle \(\psi_0\) of the perihelion changes from orbit to orbit. 
For a close-to-circular orbit such as Mercury's, the effect of GR is to modify the 
effective potential to
\begin{equation*}
\begin{split}\Phi_\mathrm{eff}(r;L) = \Phi(r) + \frac{L^2}{2\,r^2} + {GM L^2 \over c^2\,r^3}\,,\end{split}
\end{equation*}
where \(c\) is the speed of light. We can use this to compute the GR precession of the 
pericenter angle \(\psi\) assuming that the orbit overall remains close to the 
Keplerian one. Mercury has a semi-major axis of 0.387098 AU and an 
eccentricity of 0.205630.\\

(a) Using numerical integration, compute the GR precession between successive pericenter 
passages (the difference between the GR prediction and the Newtonian prediction).\\

(b) We can also estimate the GR precession analytically. There are various ways to do 
this. One consists of using Equation (5.35), expanding the solution as \(v = u-u_0\) 
around \(u_0\), the average value of \(u\) in the purely-Keplerian solution, and 
discarding terms of order \(v^2\). Then you can solve for \(u(\psi)\) and determine 
the GR precession analytically. Demonstrate that the final result is
\begin{equation*}
\begin{split}\delta \psi = {6\pi\,GM \over a\,(1-e^2)\,c^2}\,\end{split}
\end{equation*}
and check that the numerical value of this agrees with what you obtained in (a).\\

\noindent{\bf Problem 4:} In the derivation and application of the relaxation time in 
Section 6.1, we approximated galaxies as consisting of stars with mass \(m\). In 
reality, however, galaxies contain significant amounts of dark matter 
(\(\approx 50\%\) within a galaxy's visible region). Let's investigate when and how 
this matters!\\

(a) Assuming that dark matter is a new fundamental particle with a mass 
\(m_\mathrm{DM} \ll m\), what is the relaxation time when one takes the fraction of a 
galaxy's mass contain in dark matter into account?\\

(b) If dark matter is \emph{not} a fundamental particle, but is instead a heavy object, 
roughly at what \(m_\mathrm{DM}\) do we have to start accounting for the number of 
dark-matter particles in a galaxy when calculating \(t_\mathrm{relax}\)?\\

(c) Suppose all of the dark matter is \(100\,M_\odot\) primordial black holes 
(a somewhat viable dark-matter candidate; Carr et al. 2021). What is the relaxation 
time in the Milky Way in this case?\\

(d) Eridanus II is a Milky-Way satellite galaxy with a compact star cluster near its 
center that could be plausibly destroyed by two-body encounters with massive 
dark-matter particles. The cluster has a half-light radius of \(r_h = 13\,\mathrm{pc}\) 
and the central dark-matter distribution has a density 
\(\rho_\mathrm{DM} \approx 0.3\,M_\odot\,\mathrm{pc}^{-3}\) 
and velocity dispersion \(\approx 5\,\mathrm{km\,s}^{-1}\). Again assuming all of the 
dark matter is \(100\,M_\odot\) primordial black holes, estimate the two-body relaxation 
time for stars in the compact stellar cluster.

\end{document}
