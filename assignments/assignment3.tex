\documentclass[12pt]{article}
\usepackage[letterpaper,margin=1in]{geometry}
\usepackage{hyperref}
\usepackage{url}
\urldef\galpyconfig\url{https://docs.galpy.org/en/latest/installation.html#configuration-file}
\usepackage{amsmath,amssymb}
\begin{document}
\begin{center}
{\bf \LARGE AST1420 ``Galactic Structure and Dynamics'' Problem Set 3}\\[7pt]
\emph{Due on Nov. 19 at the start of class}\\[7pt]
\end{center}

Most of the exercises in this problem set must be solved on a computer
and the best way to hand in the problem set is as an \texttt{ipython
  notebook}. Rather than sending me the notebook, you can upload it to
\texttt{GitHub}, which will automatically render the notebook. Rather
than starting a repository for a single notebook, you can upload your
notebook as a \texttt{\href{https://gist.github.com/}{gist}}, which
are version-controlled snippets of code that can optionally be made
private. If you want to make further changes, you can clone your gist
in a separate directory and use it as you would any other git
repository. \emph{Please re-run the entire notebook (with \texttt{Cell
    > Run All}) after re-starting the notebook kernel before uploading
  it}; this will make sure that the input and output are fully
consistent.

If you are unfamiliar with notebooks, you can also send in a
traditional write-up (in LaTeX), but you also need to send in
well-commented code for how you solved the problems. Thus, notebooks
are strongly preferred :-)\\

\noindent{\bf Problem 1:} Gravitational collapse in one dimension.

\begin{quote}
  ``[One] grows stale if he works all the time on insoluble problems, and a trip to the beautiful world of one dimension will refresh his imagination better than a dose of LSD.''\flushright Freeman Dyson
\end{quote}

Gravitational $N$-body simulations of structure formation in the
Universe are complex and computationally demanding. Fortunately, some
of the physics of gravitational collapse can be understood by
simulating one-dimensional systems, where it is easy to reach high
resolution and where gravity is simpler. Let's explore the formation
of dark matter halos using gravitational $N$-body simulations in one
dimension.\\

(a) The Poisson equation is as usual
\begin{equation}
  \nabla^2 \Phi = 4\pi G \rho\,.
\end{equation}
An important aspect of understanding gravitation is the determination
of the \emph{Green's function}, that is, the solution of this equation
for a density $\rho(x) \propto \delta(x)$. In three dimensions, we
showed in the notes that the gravitational potential for
$\rho(\vec{x}) = M\,\delta(\vec{x})$ is the familiar $\Phi = -GM/r$
with $r = |\vec{x}|$. Show that the solution for $\rho(x) =
A\,\delta(x)$ in one dimension is given by $\Phi(x) = 2\pi
G\,A\,|x|$. What are the units of $A$?\\

(b) The gravitational force corresponding to $\Phi(x) = 2\pi
G\,A\,|x|$ is $F = -\mathrm{d} \Phi / \mathrm{d} x = -2\pi
G\,A\,\mathrm{sign}(x)$, where $\mathrm{sign}(x)$ is the sign function
that is equal to one for $x > 0$, equal to minus 1 for $x < 0$ and
equal to zero for $x=0$. Therefore, very unlike what happens in three
dimensions, the gravitational force is constant as a function of
distance! In the next part, we will run an $N$-body simulation for $N$
equal-mass particles (so all $A$ are equal; you can also assume that
$2\pi G = 1$). Because the gravitational force is constant with
distance, the total force on any given particle $i$ in the sequence is
therefore given by $A\times\,(N^+_i-N^-_i)$, where $N^+$ is the number
of particles with $x > x_i$ and $N^-_i$ is the number of particles
with $x < x_i$. Write a function that for a given array of positions
$x_i$ computes the total force on each particle. Test this function by
applying it to a large number of particles uniformly distributed
between $-1/2$ and $1/2$, for which you should compute the analytical
solution.\\

(c) Now write the second part of the $N$-body code by writing a
leapfrog integrator that integrates all $N$ particles forward for a
time step $\delta t$ using the force function that you wrote in
(b). Use this $N$-body code to integrate the following system of
initial conditions
\begin{align}
  x & \in [-\pi/2,\pi/2]\\
  v &= -0.001\,\sin(x)\\
  N & = 10,001\,,
\end{align}
where the $x$ are evenly spaced in the interval given. These initial
conditions are similar to those of a dark matter halo that has just
started collapsing after decoupling from the Hubble
expansion. Integrate this system forward for a total time of $t = 200$
with $\delta t = 0.0005$. Plot the phase-space distribution $(x,v)$ at
times $t=0, 18, 25, 40, 132$, and $200$. Describe what you see
happening.\\

(d) The simulation that you ran is the same as that shown in the GIF
at the start of chapter 6 in the notes, but the simulation in the
notes solves the $N$-body problem exactly (which is possible in 1D,
because the force does not depend on distance). Compare your
simulation's output to the GIF and discuss why your simulation might
(dis)agree with the GIF.\\

\noindent{\bf Problem 2:} Recycling in chemical evolution and the
abundance of deuterium.\\

In the class notes, we ignored the effect of recycling of unprocessed
material by stars back into the ISM. That is, we assumed that gas
consumed by star formation was fully lost from the ISM, except for the
enriched ejecta that we described using the yield parameter
$p$. However, in reality, winds from massive stars return a
significant amount of mass to the ISM that was not changed by the star
and thus returns mass to the ISM at the star's birth abundance $Z$.\\

Recycling is especially important to consider if we want to
investigate the abundance of deuterium in the ISM. Deuterium is an
interesting element because it is only \emph{destroyed} by stars
without being created (to a good approximation, deuterium is only
produced during Big Bang nucleosynthesis [BBN]). As such, deuterium is
a good tracer of whether or not gas has ever been in a star. For
example, if most of the gas in the present-day Milky Way ISM was
previously processed in stars, then the deuterium abundance in the ISM
should be very small, because all of the deuterium should have been
destroyed. The deuterium abundance of the ISM can be determined using
UV spectroscopy and it is found to be approximately $90\%$ of the
primordial BBN abundance. Let's see what we can learn about chemical
evolution from this basic observation!\\

In this problem, we will denote the mass of the ISM in deuterium as
$M_D$ and the fraction of the ISM in deuterium as $X_D = M_D/M_g$. The
primordial deuterium abundance is $X_D^P$ (you don't need to know the
actual value, but it is $\approx 2.6\times 10^{-5}$). The observations
of deuterium of the ISM show that today $X_D/X_D^P \approx 0.9$. All
of the following questions can be solved analytically using the same
techniques as used in the notes for the closed/leaky/accreting box
models.\\

(a) Extend the closed box model to include the effect of recycling,
which we will model as happening instantaneously. Assume that a
fraction $r$ of the mass turned into stars is returned to ISM at the
ISM's abundance at the time of the formation of the
star. Specifically, derive the relation between $Z$, $p$, and the gas
fraction as in Equation (13.7) in the presence of
recycling. Discuss.\\

(b) Now work out the closed box model with recycling for
deuterium. Remember that no deuterium is created by stars. What is the
relation between $X_D/X_D^P$, $r$, and the gas fraction? For the
current ISM's values of the deuterium abundance and the gas fraction,
what recycling fraction do you need to match the two?\\

(c) Stellar evolution tells us that $r \approx 0.4$. Compare this to
the value you found in (b). If they are significantly different,
explain the physical reason for this in the context of the closed box
model.\\

(d) In class, we discussed how the accreting box model is successful
at explaining the absence of a large number of very metal poor stars
in the solar neighborhood (the G dwarf problem). Work out the
evolution of the deuterium abundance $X_D/X_D^P$ in the accreting box
model (with $\eta=0$), remembering that the specific model we looked
at has a constant gas mass and that any inflowing gas has the
primordial deuterium abundance. What recycling fraction $r$ do you
need now to match $X_D/X_D^P$ given the observed gas fraction?\\

(e) Write the relation that you found in (d) in terms of the
metallicity $Z$ of the ISM rather than the total-to-gas ratio
$M/M_g$. Note that for this you will need to adjust the notes'
accreting box model to take into account recycling (again use the
$\eta=0$ version to keep things simpler). Try to obtain a \emph{very}
simple relation using the fact that $Z_\odot/p \ll 1$. Fun fact: the
relation that you derive actually holds very generally, so there is a
direct relation between the metallicity of the ISM and the ratio
$X_D/X_D^P$, regardless of the details of inflow and outflow and the
star formation history.\\

One take-away from this problem should be that most of the hydrogen in
the disk of the Milky Way has \emph{never} been inside of a
star. While all of the heavy elements in your body were produced in
stars, the hydrogen atoms that make up the $70\%$ or so of water in
your body were actually created in the Big Bang!\\

\end{document}
