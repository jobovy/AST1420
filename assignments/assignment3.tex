\documentclass[12pt]{article}
\usepackage[letterpaper,margin=1in]{geometry}
\usepackage{hyperref}
\usepackage{amsmath}
\usepackage{amssymb}
\begin{document}
\begin{center}
{\bf \LARGE AST1420 ``Galactic Structure and Dynamics'' Problem Set 3}\\[7pt]
\emph{Due on Nov. 13 at the start of class}\\[7pt]
\end{center}

Most of the exercises in this problem set must be solved on a computer
and the best way to hand in the problem set is as an \texttt{jupyter
  notebook}. Rather than sending me the notebook, you can upload it to
\texttt{GitHub}, which will automatically render the notebook. Rather
than starting a repository for a single notebook, you can upload your
notebook as a \texttt{\href{https://gist.github.com/}{gist}}, which
are version-controlled snippets of code that can optionally be made
private.

If you want to upload your notebook as a gist from the command-line,
you can use the package \href{http://github.com/defunkt/gist}{at this
  http URL} and use it as follows. Log into your \texttt{GitHub}
account:\\

\texttt{gist --login}\\

and then upload your notebook
\texttt{AST1420\_2018\_PS3\_YOURNAME.ipynb} as\\

\texttt{gist -p AST1420\_2018\_PS3\_YOURNAME.ipynb}\\

(the \texttt{-p} option will make the gist private). If you want to
make further changes, you can clone your gist in a separate directory
and use it as you would any other git repository. \emph{Please re-run
  the entire notebook (with \texttt{Cell > Run All}) after re-starting
  the notebook kernel before uploading it}; this will make sure that
the input and output are fully consistent. 

If you are unfamiliar with notebooks, you can also hand in a
traditional write-up, but you also need to send in well-commented code
for how you solved the problems. Thus, notebooks are strongly
preferred :-)\\

\noindent{\bf Problem 1:} (10 points) Growth of perturbations in the
Universe. We discussed the growth of dark matter and baryonic
perturbations in class and in the notes; here we dig a little deeper
into how these perturbations evolve.\\

\noindent{\bf (a)} We argued that as the Universe transitions from
being matter dominated at $z \gtrsim 1$ to being dark-energy dominated
at $z \lesssim 0.5$ perturbations in the linear regime stop
growing. Show this expliticly by computing and plotting the growing
mode for our Universe (which you can approximate as consisting of dark
energy and dark matter with density parameters today of
$\Omega_{\Lambda,0} = 0.7$ and $\Omega_{m,0} = 0.3$) from $a = 10^{-3}$
to $a=10$ and compare it to growth in a matter-dominated
Universe. Normalize the growing mode such that $\delta_m(a=1) =
1$. Also overplot the scale factor at which dark energy and dark
matter have equal densities. Discuss.\\

\noindent{\bf (b)} We discussed how on large scales---scales much
larger than the Jeans scale---baryonic perturbations follow the same
differential equation as dark matter perturbations. However, this does
not directly imply that baryonic perturbations and dark matter
perturbations are equal, because their initial conditions differ. In
particular, before recombination, baryons are coupled to photons,
undergo baryon acoustic oscillations, and they therefore do not grow,
while dark matter can already start growing. To investigate what
happens, consider the evolution equation for baryons in the limit of
zero pressure and using the approximation that the gravitational
potential is contributed by the \emph{dark matter} only. Directly
integrate the differential equation (you can used
\texttt{scipy.odeint} for this) starting with zero baryon perturbation
and velocity at $t=10^{-4.5}$ in an Einstein--de-Sitter Universe
(where $t=1$ corresponds to $a=1$) and integrate until $t=100$ (you
want to use a logarithmic time grid). Compare the solution to that of
the dark matter. What happens?\\

\noindent{\bf (c)} On scales smaller than the Jeans scale, baryonic
perturbations cannot grow. Again working in the limit that the
gravitational potential is sourced by the dark matter alone, show that
in an Einstein--de-Sitter Universe where the baryon equation of state
is $P \propto \rho^{4/3}$ a solution of the equation is given by

\begin{equation}
  \delta_b(\vec{k},t) = \frac{\delta_c(\vec{k},t)}{1+k^2/k_J^2}\,,
\end{equation}

where $k_J$ is the comoving Jeans wavenumber and $k = |\vec{k}|$. Plot
$\delta_b/\delta_c$ as a function of $k$ (you will have to compute the
sound speed for this, which you can do using the knowledge that at
$z\approx 1,000$ the temperature was $\approx
3,000\,\mathrm{K}$). Compare this to what this function looks like
before recombination, when baryons are coupled to photons. Discuss.\\

\noindent{\bf (d)} (extra credit) A more realistic equation of state
is that of a mono-atomic gas. Still working in the Einstein--de-Sitter
Universe with the approximation that dark matter alone sources the
gravitational potential, explicitly integrate the differential
equation for $\delta_b(\vec{k},t)$ from $a = 10^{-3}$ to $a = 10$
starting from a vanishing density perturbation and plot the resulting
function as a function of time for a few well-chosen k and as a
function of $k$ for a few well-chosen times. Compare to the behavior
you found in (c) and discuss in terms of what you saw in (b) and
(c). Hint: To check your code, it's a good idea to test that you
recover the result in (c). For obtaining numerical values, use that
the Hubble constant is $H_0 =
70\,\mathrm{km\,s}^{-1}\,\mathrm{Mpc}^{-1}$.\\

\noindent{\bf Problem 2:} (10 points) Chaos in the Milky Way? In
class, we discussed the Henon \& Heiles (1964) potential and how it
exhibits chaotic regions. Let's see whether more realistic galactic
potentials have chaotic regions!\\

\noindent{\bf (a)} An older, but useful for this exercise, model for
the Milky Way potential was presented by Helmi (2004; MNRAS {\bf 351},
643). This model consists of a Hernquist bulge, a Miyamoto-Nagai disk,
and a logarithmic halo with a core radius. These are all basic
\texttt{galpy} potentials. Setup this model in \texttt{galpy}, plot
the rotation curve, and compare to the top panel of Fig. 1 of Helmi
(2004).\\

\noindent{\bf (b)} Let's investigate orbits in this potential. Before
continuing, it's easiest to turn off the physical outputs of the
potential that you setup in (a):
\[\texttt{galpy.potential.turn\_physical\_off(YOUR\_POTENTIAL\_LIST)}.\] Write
functions that (i) compute the energy $E_c(R_c)$ of a circular orbit
at a given radius $R_c$ and (ii) the $z$-component of the angular
momentum $L_c(E_c)$ for this orbit. What is the energy and angular
momentum of a circular orbit at $R_c = 30\,\mathrm{kpc}$? Use the
function from the notes to setup an orbit with a given radius, radial
velocity, energy, and angular momentum. Check that you are correctly
computing the energy and angular momentum for the circular orbit with
$R_c = 30\,\mathrm{kpc}$ by setting up an orbit using this function,
integrating it, and displaying it in $(x,y)$ and $(R,z)$.\\

\noindent{\bf (c)} Investigate the surface of section $(R,v_R)$ (using
$z=0$ as the surface at which to record points) for orbits with $E =
E_c(R_c = 30\,\mathrm{kpc})$ and $L_z = 0.15\,L_c(E_c)$ in a similar
manner to how we did this in the notes in Chapter 11. Try to find
interesting parts of the surface of section and discuss these.\\

\noindent{\bf (d)} Repeat the analysis in (c) for orbits closer to the
solar neighborhood, at $R_c=9\,\mathrm{kpc}$ (again using $L_z =
0.15\,L_c(E_c)$). Again try to find interesting parts of the surface
of section and discuss what you find.\\

\noindent{\bf (e)} Repeat the analysis in (c), but for a prolate halo:
that is, set the logarithmic halo's flattening parameter to $q=1.25$.
Again try to find interesting parts of the surface of section and
discuss what you find.\\

Note for all of the above: like in the notes, to trace an orbit in the
surface of section, you have to integrate for a long time and obtain
high time resolution. A time sequence like \texttt{ts=
  numpy.linspace(0.,10000.,1000001)} appears to work okay, but you
might have to adjust this for some orbits.

\end{document}
