\documentclass[12pt]{article}
\usepackage[letterpaper,margin=1in]{geometry}
\usepackage{hyperref}
\usepackage{amsmath,amssymb}
\begin{document}
\begin{center}
{\bf \LARGE AST1420 ``Galactic Structure and Dynamics'' Problem Set 3}\\[7pt]
\emph{Due on Mar. 27 by 5pm}\\[7pt]
\end{center}

Some of the exercises in this problem set must be solved on a computer
and a good way to hand in the problem set is as a \texttt{jupyter
  notebook}. \emph{Please re-run the entire notebook (with \texttt{Cell
    > Run All}) after re-starting the notebook kernel before sending
  it in}; this will make sure that the input and output are fully
consistent. You can also send in a traditional write-up in LaTeX as a PDF, 
but then you also need to send in well-commented code for how you solved 
the numerical problems. Thus, notebooks are preferred :-)\\

\noindent{\bf Problem 1:} In the accreting box models considered in Chapter 12, it is 
assumed that infalling gas is pristine, that is, that it has zero metals in it. However, 
in reality infalling gas likely has some small, non-zero abundance \(Z_{\mathrm{in}}\). 
Let's explore how this affects the evolution of the accreting box models that we have 
discussed.\\

(a) In the original accreting box model (without outflows or recycling), show that the 
effect of enriched inflows is such that the evolution remains the same if the yield 
parameter \(p\) is replaced by \(p+Z_{\mathrm{in}}\).\\

(b) Now also consider outflows and recycling as done in the text. Demonstrate that the 
evolution remains the same as long as we replace 
\(p \rightarrow p\,(1+Z_{\mathrm{in}}/Z_{\mathrm{eq}})\), where \(Z_{\mathrm{eq}}\) 
is the equilibrium abundance of the model.\\

The relation you derived in b.\ remains valid even if we allow the gas supply to slowly 
decline or increase over time. Because \(Z_{\mathrm{in}}\) is typically much less than 
\(Z_{\mathrm{eq}}\), this shows that enriched inflows only have a very minor effect on 
the chemical evolution of a galaxy, except early in its evolution when the abundance of 
the ISM is low and the gas supply might be increasing rapidly.\\

\noindent{\bf Problem 2:} The abundance of deuterium in the ISM. Deuterium is an 
interesting element, because it is only \emph{destroyed} whenever it 
enters a star, with no deuterium created or recycled by stars as part of normal or 
explosive stellar evolution. Thus, to a good approximation, deuterium is only produced 
during Big Bang nucleosynthesis (BBN). As such, deuterium is a good tracer of whether 
or not gas has ever been in a star. For example, if most of the gas in the present-day 
Milky Way ISM was previously processed in stars, then the deuterium abundance in the 
ISM should be very small, because all of the deuterium should have been destroyed. 
The deuterium abundance of the ISM can be determined using UV spectroscopy and it is 
found to be approximately 90\% of the primordial BBN abundance. Let’s see what we can 
learn about chemical evolution from this basic observation!\\

In this problem, we denote the mass of the ISM in deuterium as \(M_D\) and the fraction 
of the ISM in deuterium as \(X_D = M_D/M_g\). The primordial deuterium abundance is 
\(X_D^P\) (you don't need to know the actual value, but it is 
\(\approx 2.6\times 10^{-5}\)). The observations of deuterium of the ISM show that 
today \(X_D/X_D^P \approx 0.9\). All of the following questions can be solved 
analytically using the same techniques as used in Chapter 12 for the 
closed/leaky/accreting box models.\\

(a) Work out the closed box evolution with recycling for deuterium. What is the 
relation between \(X_D/X_D^P\), \(r\), and the gas fraction? For the current ISM's 
values of the deuterium abundance and the gas fraction, what recycling fraction do you 
need to match the two? As discussed in the text, stellar evolution tells us that 
\(r \approx 0.4\). Compare this to the value that you find in the closed box model. 
If they are significantly different, explain the physical reason for this in the context 
of the closed box model.\\

(b) Work out the evolution of the deuterium abundance \(X_D/X_D^P\) in the accreting box 
model without outflows, remembering that the specific model that we discussed in the 
text has a constant gas mass and that any inflowing gas has the primordial deuterium 
abundance. What recycling fraction \(r\) do you need now to match \(X_D/X_D^P\) given 
the observed gas fraction? Similar to how we argued that the long-term 
steady state of the accreting box model has \(Z=p\), there is a simple argument for the 
result that you find.\\

(c) Repeat b., but including the effect of outflows with constant \(\eta\).\\

(d) Write the relations that you found in b. and c. in terms of the metallicity \(Z\) 
of the ISM rather than the total-to-gas ratio \(M/M_g\). Try to obtain a \emph{very} 
simple relation using the fact that \(Z_\odot/p \ll 1\). Fun fact: the relation that you 
derive actually holds very generally, so there is a direct relation between the 
metallicity of the ISM and the ratio \(X_D/X_D^P\), regardless of the details of 
inflow and outflow and the star formation history (Weinberg 2017; ApJ, 851, 25).\\

One take-away from this problem is that the deuterium abundance in the Milky Way's ISM 
is another basic observation (like the G dwarf problem and the low abundance of the 
present-day Milky Way ISM compared to stellar yields) that is naturally explained by 
including inflows and outflows of gas in galaxies. Another take-away is that most of the 
hydrogen in the disk of the Milky Way has \emph{never} been inside of a star. While all 
of the heavy elements in your body were produced in stars, the hydrogen atoms that make 
up the 70\% or so of water in your body were actually created in the Big Bang!\\

\noindent{\bf Problem 3:} While the derivation of the halo mass function in Chapter 18 
does not mention it, the spherical-collapse framework in which it 
is calculated implies that the \emph{halo mass} in the halo mass function is the mass 
contained within the virial radius determined using the standard \(\Delta_v\) 
(e.g., Equation 18.133 for an Einstein--de Sitter Universe). However, both in 
simulations and observations, halo masses are sometimes defined using a different virial 
overdensity \(\Delta_v\). This leads to a different mass for the same object and, thus,
the halo mass function shifts. Because dark-matter halos are well 
described by an NFW profile with a tight concentration\textendash{}mass relation, we can 
use the NFW profile to translate between different mass definitions. Assuming that the 
basic halo mass function's \(\Delta_v = 18\pi^2\) (e.g., the Sheth-Tormen halo mass 
function was explicitly calibrated using this overdensity), determine and plot the 
present-day halo mass function for \(\Delta_v = 18\pi^2, 200, 400, 800, 1600\), using 
the Dutton \& Maccio (2014) concentration--mass relation from Equation (3.79) to obtain 
the halo's inner density profile at a given mass. Discuss the trends that you see.

\end{document}
