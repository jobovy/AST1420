\documentclass[12pt]{article}
\usepackage[letterpaper,margin=1in]{geometry}
\usepackage{hyperref}
\usepackage{amsmath}
\usepackage{amssymb}
\begin{document}
\begin{center}
{\bf \LARGE AST1420 ``Galactic Structure and Dynamics'' Problem Set 2}\\[7pt]
\emph{Due on Oct. 23 at the start of class}\\[7pt]
\end{center}

Most of the exercises in this problem set must be solved on a computer
and the best way to hand in the problem set is as an \texttt{jupyter
  notebook}. Rather than sending me the notebook, you can upload it to
\texttt{GitHub}, which will automatically render the notebook. Rather
than starting a repository for a single notebook, you can upload your
notebook as a \texttt{\href{https://gist.github.com/}{gist}}, which
are version-controlled snippets of code that can optionally be made
private.

If you want to upload your notebook as a gist from the command-line,
you can use the package \href{http://github.com/defunkt/gist}{at this
  http URL} and use it as follows. Log into your \texttt{GitHub}
account:\\

\texttt{gist --login}\\

and then upload your notebook
\texttt{AST1420\_2018\_PS2\_YOURNAME.ipynb} as\\

\texttt{gist -p AST1420\_2018\_PS2\_YOURNAME.ipynb}\\

(the \texttt{-p} option will make the gist private). If you want to
make further changes, you can clone your gist in a separate directory
and use it as you would any other git repository. \emph{Please re-run
  the entire notebook (with \texttt{Cell > Run All}) after re-starting
  the notebook kernel before uploading it}; this will make sure that
the input and output are fully consistent. 

If you are unfamiliar with notebooks, you can also hand in a
traditional write-up, but you also need to send in well-commented code
for how you solved the problems. Thus, notebooks are strongly
preferred :-)\\



\noindent{\bf Problem 1:} (10 points) Bertrand's theorem and the mass
distribution in the center of the Milky Way. Betrand's theorem states
that the only mass distributions for which all orbits close are the
(a) point-mass and (b) homogeneous density sphere. Let's investigate
this here and see what it implies about the mass distribution in the
Galactic center.\\

\noindent{\bf (a)} If all orbits in a mass distribution close, then in
particular orbits that are close to a circular orbit must
close. Close-to-circular orbits are described by the epicycle
approximation, so we can use this approximation to see whether orbits
close. Using the epicycle approximation for spherical mass
distributions---the same as that discussed for disks, but without the
vertical dependence of the potential---demonstrate that the only
gravitational potentials for which close-to-circular orbits close have
$\phi(r) \propto r^{\beta^2-2}$ where $\beta$ is a rational number
(you can start from the assumption that the potential has a power-law
form, because over a small range of radii, all potentials can be
approximated as such).\\

\noindent{\bf (b)} The form $\phi(r) \propto r^{\beta^2-2}$ with
$\beta$ a rational number includes potentials where $\beta=5/3$ or
$\beta=4/3$. Through explicit orbit integration using \texttt{galpy},
investigate whether all orbits close in potentials with $\beta =
[1,16/15,5/4,4/3,3/2,11/6,23/12,2]$. What can you conclude?\\

\noindent{\bf (c)} Now numerically show that what you claim in (b) is
correct by explicitly calculating the radial and azimuthal periods of
well-chosen orbits in each of these potentials.\\

\noindent{\bf (d)} In the Galactic center, we can observe (partial)
orbits of the so-called ``S stars''. In particular, the orbit of the
star S2 (or S0-2 depending on who you ask) has been observed through a
full azimuthal period and its orbit closes to within the
uncertainties. While there are closed orbits in many potentials, the
fact that the one orbit that we observe closes (and which is not
circular) is good evidence that \emph{all} orbits close in the mass
distribution that S2 is orbiting in. From Bertrand's theorem we know
that this means that the mass distribution is dominated either by a
massive point-like object or that it is homogeneous. These are quite
different mass distributions! Discuss observations of the S stars that
distinguish between these two possibilities.\\

\noindent{\bf Problem 2:} (10 points) Deviations from circular
rotation in external galaxies. In chapter 9, we derived the
two-dimensional velocity fields of external galaxies for various
rotation curves. In this derivation, we assumed that the gas in these
galaxies is on circular orbits. In this problem, we will explore what
happens when this is not the case and what observational limits we can
place on deviations from circular motion.\\

\noindent{\bf (a)} Following the derivation in Chapter 9.1.2, derive
the expression for $V(x,y)$ when gas also has a radial component
$v_R(R)$. For simplicity, assume the same geometry as in Chapter 9.1.2
and that the radial component does not depend on $\theta$.\\

\noindent{\bf (b)} Using this expression, discuss what happens to the
two-dimensional velocity field when the rotation curve is that of
solid-body rotation, but $v_R(R)$ is some constant fraction $f=0.1$ of
$v_c(R)$ [that is, $v_R(R) = f\,v_c(R)$].\\

\noindent{\bf (c)} How does the behavior you see in (b) generalize to
other rotation curves? That is, what is the relation between the
two-dimensional velocity field for $f=0$ and for general $f$? When you
have photometry (or HI emission) \emph{and} kinematics for an external
galaxy (like for NGC 3198 in the notes), how can you combine these
observations to test for $f \neq 0$?\\

\noindent{\bf (d)} Plot the two-dimensional velocity field for the
three basic cases considered in the notes (solid-body rotation, flat
rotation curve, peaked rotation curve) for $f=0.1$. Discuss what
happens in terms of (c) above.\\

\noindent{\bf (e)} Use the HI emission and two-dimensional velocity
field of NGC 3198 from Walter et al. (2008; see notes) and the above
behavior to estimate an upper limit on $f$ for this galaxy (without
having the actual data in hand, simply provide a rough estimate based
on the figure in the notes).\\

\end{document}
