\documentclass[12pt]{article}
\usepackage[letterpaper,margin=1in]{geometry}
\usepackage{hyperref}
\usepackage{amsmath,amssymb}
\begin{document}
\begin{center}
{\bf \LARGE AST1420 ``Galactic Structure and Dynamics'' Problem Set 2}\\[7pt]
\emph{Due on Feb. 28 by 5pm}\\[7pt]
\end{center}

Some of the exercises in this problem set must be solved on a computer
and a good way to hand in the problem set is as a \texttt{jupyter
  notebook}. \emph{Please re-run the entire notebook (with \texttt{Cell
    > Run All}) after re-starting the notebook kernel before sending
  it in}; this will make sure that the input and output are fully
consistent. You can also send in a traditional write-up in LaTeX as a PDF, 
but then you also need to send in well-commented code for how you solved 
the numerical problems. Thus, notebooks are preferred :-)\\

\noindent{\bf Problem 1:} The density flattening of the logarithmic potential.\\

(a) Equation (8.24) for the density flattening of a flattened logarithmic potential with 
potential flattening \(q\) is derived by defining the density flattening \(q_\rho\) as 
the ratio \(z_m/R_m\) of the values \(z_m\) and \(R_m\) at which a constant-density 
contour intersects the \(z\) and \(R\) axes, respectively (e.g., for a contour with 
\(\rho(R,z) = C\), \(z_m\) is the value for which \(\rho(0,z_m) = C\)). Derive Equation 
(8.24).\\

(b) A different way of defining the flattening is to locally approximate the 
constant-density contour as an ellipse in \((R,z)\) and using that ellipse's axis ratio 
as the flattening. Demonstrate that the density flattening in this case is given by
\begin{equation*}
\begin{split}q_\rho^2 = {z\over R}\,{\partial \rho(R,z) / \partial R \over \partial \rho(R,z) / \partial z}\,.\end{split}
\end{equation*}

(c) For a representative set of \(q\), determine the flattening along a constant-density 
contour using the definition in b. and compare it to Equation (8.24). Discuss your results.\\

\noindent{\bf Problem 2:} We did a rotation-curve activity with the SPARC sample of 175 
nearby galaxies with mid-infrared photometry photometry from the Spitzer space telescope 
(at \(3.6\,\mu\mathrm{m}\)) and precise rotation curves from 21cm and H\(\alpha\) 
observations in class (these are also used in Chapter 9.6). Let's see what else we can 
learn from this sample!\\

(a) In the activity, we tried to fit the rotation curves with exponential disks plus 
NFW halos without using the observed surface brightness profiles. We saw that there was 
a significant degeneracy between the disk and halo parameters. However, we can use the
observed surface brightness profiles to break this degeneracy. When we restrict 
ourselves to the sub-sample of pure disk galaxies (i.e., without a bulge component, 
which are galaxies with very small \texttt{Vbul} in the database), a simple exponential 
surface-brightness profile provides a good fit. For the galaxies UGC 05918, UGC 08490, and 
NGC 6503, find good disk parameters by fitting the surface brightness profile with an 
exponential and then predict the disk rotation curve assuming \(M/L=1\), as is 
appropriate in the mid-infrared. What do you see?\\

(b) Now fit exponential-disk + NFW-halo models to the rotation curves by keeping the 
disk model fixed at that from (a) and adjusting the NFW parameters by hand to obtain a 
good fit to the rotation curve. What do you learn about the dark matter contribution in 
these galaxies?\\

\noindent{\bf Problem 3:} Chaos in a popular model for galactic disks? A useful way to 
visualize orbits in axisymmetric and non-axisymmetric potentials is to use surfaces of
section. We did not discuss these in class, so we wil use this exercise as an excuse to 
learn about them! Briefly, for a time-independent axisymmetric potential, a surface of 
section is a 2D surface in the 4D phase space \((R,v_R,z,v_z)\) of an orbit. Because of 
the conservation of energy, the orbit is confined to a 3D surface in this 4D space, and
when we slice this 3D surface with a 2D surface of section, we get a curve in 2D. In the 
absence of a third integral of motion, this curve should fill a 2D area when one 
integrates long enough and the orbit is chaotic. But if there is a third isolating 
integral of motion, then the curve traces out a simple 1D curve. Thus, we can use 
surfaces of section to investigate whether orbits in a given potential are regular or 
chaotic. For an axisymmetric potential that is symmetric with respect to the \(z=0\) 
plane, we use the $z=0$ plane as the surface, only considering orbits with \(v_z > 0\). 
Read Sections 14.1, 14.2, and 14.3 of the book to learn more about surfaces of section 
and feel free to liberally use code you find there to do this problem.\\

As we discussed in Chapter 8.2.2, a simple yet not-that-unrealistic model for the 
gravitational potential of galactic disks is the Miyamoto-Nagai model (see Equation 8.19). 
This model is a thickened version of the Kuzmin disk model from Chapter 8.2.1. We will 
not demonstrate this here, but an interesting property of the Kuzmin disk is that all
orbits are regular (i.e., not chaotic and satisfy a third integral of motion in addition 
to $E$ and $L_z$). There is no known analytical way to determine whether or not all 
orbits in Miyamoto-Nagai disks are regular, but we can investigate numerically using 
surfaces of section.\\

(a) To conveniently initialize potentially chaotic orbits in different potentials, write 
functions that (i) compute the energy \(E_c(R_c)\) of a circular orbit at a given radius 
\(R_c\) in a given potential \(\Phi\) and (ii) the \(z\)-component of the angular momentum 
\(L_c(E_c)\) for this orbit.\\

(b) Orbits with low \(L_z/L_c\) have large eccentricities and are most likely to be 
chaotic. For a Miyamoto-Nagai disk model with \(a=1\) and \(b/a = 0.1\), investigate 
the orbit of section $(R,v_R)$ for orbits with \(E = E_c(R_c = 3\,(a+b))\) and 
\(L_z = 0.1\,L_c(E_c)\). The orbit of section in this case is a representative sampling 
of all possible orbits at this energy and angular momentum like in Figure 14.3. Does it 
show any sign of chaos?\\

(c) Repeat the analysis in b. for a model with \(b/a = 1\). Does the surface of section 
contain any chaotic orbits?\\

(d) Investigate some further surfaces of section at different \(R_c\) and \(L_z/L_c\) for 
the potentials in b. and c. Do you see any signs of chaos? The reason that you most likely 
do not is that the Miyamoto-Nagai model smoothly goes between the Kuzmin 
model (for \(b/a=0\)) to the Plummer model (for \(a/b=0\)), both of which have all 
regular orbits (for the Plummer model this is because it is spherical and all orbits in 
spherical potentials are regular). This appears to be a strong enough constraint to 
avoid chaos in all Miyamoto-Nagai models (although this has not been 
proven and perhaps you have found a chaotic orbit!).

\end{document}
