\documentclass[12pt]{article}
\usepackage[letterpaper,margin=1in]{geometry}
\usepackage{hyperref}
\usepackage{amsmath}
\usepackage{amssymb}
\begin{document}
\begin{center}
{\bf \LARGE AST1420 ``Galactic Structure and Dynamics'' Problem Set 3}\\[7pt]
\emph{Due on Nov. 17 at the start of class}\\[7pt]
\end{center}

Most of the exercises in this problem set must be solved on a computer
and the best way to hand in the problem set is as an \texttt{jupyter
  notebook}. Rather than sending me the notebook, you can upload it to
\texttt{GitHub}, which will automatically render the notebook. Rather
than starting a repository for a single notebook, you can upload your
notebook as a \texttt{\href{https://gist.github.com/}{gist}}, which
are version-controlled snippets of code that can optionally be made
private.

If you want to upload your notebook as a gist from the command-line,
you can use the package \href{http://github.com/defunkt/gist}{at this
  http URL} and use it as follows. Log into your \texttt{GitHub}
account:\\

\texttt{gist --login}\\

and then upload your notebook
\texttt{AST1420\_2017\_PS3\_YOURNAME.ipynb} as\\

\texttt{gist -p AST1420\_2017\_PS3\_YOURNAME.ipynb}\\

(the \texttt{-p} option will make the gist private). If you want to
make further changes, you can clone your gist in a separate directory
and use it as you would any other git repository. \emph{Please re-run
  the entire notebook (with \texttt{Cell > Run All}) after re-starting
  the notebook kernel before uploading it}; this will make sure that
the input and output are fully consistent. 

If you are unfamiliar with notebooks, you can also hand in a
traditional write-up, but you also need to send in well-commented code
for how you solved the problems. Thus, notebooks are strongly
preferred :-)\\

\noindent{\bf Problem 1:} (10 points) Deviations from circular
rotation in external galaxies. In chapter 7, we derived the
two-dimensional velocity fields of external galaxies for various
rotation curves. In this derivation, we assumed that the gas in these
galaxies is on circular orbits. In this problem, we will explore what
happens when this is not the case and what observational limits we can
place on deviations from circular motion.\\

\noindent{\bf (a)} Following the derivation in Chapter 7.1.2, derive
the expression for $V(x,y)$ when gas also has a radial component
$v_R(R)$. For simplicity, assume the same geometry as in Chapter 7.1.2
and that the radial component does not depend on $\theta$.\\

\noindent{\bf (b)} Using this expression, discuss what happens to the
two-dimensional velocity field when the rotation curve is that of
solid-body rotation, but $v_R(R)$ is some constant fraction $f=0.1$ of
$v_c(R)$ [that is, $v_R(R) = f\,v_c(R)$].\\

\noindent{\bf (c)} How does the behavior you see in (b) generalize to
other rotation curves? That is, what is the relation between the
two-dimensional velocity field for $f=0$ and for general $f$? When you
have photometry (or HI emission) \emph{and} kinematics for an external
galaxy (like for NGC 3198 in the notes), how can you combine these
observations to test for $f \neq 0$?\\

\noindent{\bf (d)} Plot the two-dimensional velocity field for the
three basic cases considered in the notes (solid-body rotation, flat
rotation curve, peaked rotation curve) for $f=0.1$. Discuss what
happens in terms of (c) above.\\

\noindent{\bf (e)} Use the HI emission and two-dimensional velocity
field of NGC 3198 from Walter et al. (2008; see notes) and the above
behavior to estimate an upper limit on $f$ for this galaxy (without
having the actual data in hand, simply provide a rough estimate based
on the figure in the notes).\\

\noindent{\bf Problem 2:} (10 points) Chaos in the Milky Way? In
class, we discussed the Henon \& Heiles (1964) potential and how it
exhibits chaotic regions. Let's see whether more realistic galactic
potentials have chaotic regions!\\

\noindent{\bf (a)} An older, but useful for this exercise, model for
the Milky Way potential was presented by Helmi (2004; MNRAS {\bf 351},
643). This model consists of a Hernquist bulge, a Miyamoto-Nagai disk,
and a logarithmic halo with a core radius. These are all basic
\texttt{galpy} potentials. Setup this model in \texttt{galpy}, plot
the rotation curve, and compare to the top panel of Fig. 1 of Helmi
(2004).\\

\noindent{\bf (b)} Let's investigate orbits in this potential. Before
continuing, it's easiest to turn off the physical outputs of the
potential that you setup in (a):
\[\texttt{galpy.potential.turn\_physical\_off(YOUR\_POTENTIAL\_LIST)}.\] Write
functions that (i) compute the energy $E_c(R_c)$ of a circular orbit
at a given radius $R_c$ and (ii) the $z$-component of the angular
momentum $L_c(E_c)$ for this orbit. What is the energy and angular
momentum of a circular orbit at $R_c = 30\,\mathrm{kpc}$? Use the
function from the notes to setup an orbit with a given radius, radial
velocity, energy, and angular momentum. Check that you are correctly
computing the energy and angular momentum for the circular orbit with
$R_c = 30\,\mathrm{kpc}$ by setting up an orbit using this function,
integrating it, and displaying it in $(x,y)$ and $(R,z)$.\\

\noindent{\bf (c)} Investigate the surface of section $(R,v_R)$ (using
$z=0$ as the surface at which to record points) for orbits with $E =
E_c(R_c = 30\,\mathrm{kpc})$ and $L_z = 0.15\,L_c(E_c)$ in a similar
manner to how we did this in the notes in Sections 9.2 and 9.3. Try to
find interesting parts of the surface of section and discuss these.\\

\noindent{\bf (d)} Repeat the analysis in (c) for orbits closer to the
solar neighborhood, at $R_c=9\,\mathrm{kpc}$ (again using $L_z =
0.15\,L_c(E_c)$). Again try to find interesting parts of the surface
of section and discuss what you find.\\

\noindent{\bf (e)} Repeat the analysis in (c), but for a prolate halo:
that is, set the logarithmic halo's flattening parameter to $q=1.25$.
Again try to find interesting parts of the surface of section and
discuss what you find.\\

Note for all of the above: like in the notes, to trace an orbit in the
surface of section, you have to integrate for a long time and obtain
high time resolution. A time sequence like \texttt{ts=
  numpy.linspace(0.,10000.,1000001)} appears to work okay, but you
might have to adjust this for some orbits.

\end{document}
