\documentclass[12pt]{article}
\usepackage[letterpaper,margin=1in]{geometry}
\usepackage{hyperref}
\begin{document}
\begin{center}
{\bf \LARGE AST1420 ``Galactic Structure and Dynamics'' Problem Set 1}\\[7pt]
\emph{Due on Oct. 6 at the start of class}\\[7pt]
\end{center}

Most of the exercises in this problem set must be solved on a computer
and the best way to hand in the problem set is as an \texttt{ipython
  notebook}. Rather than sending me the notebook, you can upload it to
\texttt{GitHub}, which will automatically render the notebook. Rather
than starting a repository for a single notebook, you can upload your
notebook as a \texttt{\href{https://gist.github.com/}{gist}}, which
are version-controlled snippets of code that can optionally be made
private.

If you want to upload your notebook as a gist from the command-line,
you can use the package \href{http://github.com/defunkt/gist}{at this
  http URL} and use it as follows. Log into your \texttt{GitHub}
account:\\

\texttt{gist --login}\\

and then upload your notebook
\texttt{AST1420\_2017\_PS1\_YOURNAME.ipynb} as\\

\texttt{gist -p AST1420\_2017\_PS1\_YOURNAME.ipynb}\\

(the \texttt{-p} option will make the gist private). If you want to
make further changes, you can clone your gist in a separate directory
and use it as you would any other git repository. \emph{Please re-run
  the entire notebook (with \texttt{Cell > Run All}) after re-starting
  the notebook kernel before uploading it}; this will make sure that
the input and output are fully consistent.

If you are unfamiliar with notebooks, you can also hand in a
traditional write-up, but you also need to send in well-commented code
for how you solved the problems. Thus, notebooks are strongly preferred :-)\\

\noindent{\bf Problem 1:} The Hernquist vs. the NFW models. Both of
these are popular models for the density profile of dark matter halos
(the NFW profile somewhat more so than the Hernquist profile). Let's
compare the two in some more detail.\\

\noindent{\bf (a)} We have parameterized both models in terms of
$(\rho_0,a)$, but a different parameterization is the radius
$r_{\mathrm{max}}$ and velocity $v_{\mathrm{max}}$ where the circular
velocity peaks. For the Hernquist and NFW profiles, what is
$r_\mathrm{max}/a$? And what is $v_\mathrm{max}$ for a given
$(\rho_0,a)$? Try to get as far as possible analytically, but you will
need to solve some of the necessary equations numerically. Verify your
derived relations by overplotting them over the example rotation
curves from the notes where both models are normalized to have the
same mass at $r=12a$.\\

\noindent{\bf (b)} Plot the effective potential for both the Hernquist
and NFW models normalized to have the same mass at $r=12a$ for an
orbit with angular momentum that is $1/3$ of the angular momentum of a
circular orbit at $r=12a$. What is the smallest pericenter radius that
\emph{any} bound orbit in the Hernquist model can have? What is the
relation between the pericenter radii of orbits with this angular
momentum in the Hernquist and the NFW models?\\

\newpage
\noindent{\bf Problem 2:} (Inspired by BT08 problem 3.6) Let's
consider what happens when we add energy to an orbit in a spherical
potential.\\

\noindent{\bf (a)} Consider an orbit in a spherical isochrone
potential with $b=1$ (pick an orbit that explores $r \approx 1$ and is
not too close to circular). Using orbit integration in \texttt{galpy},
add an instantaneous velocity offset when the orbit is at its
pericenter radius. Investigate what happens to the pericenter radius
of the resulting orbit. Is it larger or smaller than the original
pericenter radius? It is useful to consider the special cases where
(i) you only change the radial velocity and (ii) you only change the
tangential velocity (or equivalently the angular momentum). Does the
answer change if you consider different orbits?\\

\noindent{\bf (b)} Argue why the behavior you saw in part (a) is true
for \emph{any} orbit in \emph{any} spherical potential. (Hint:
consider the special cases and what happens to the effective
potential). You can illustrate your argument with the orbit(s) that
you investigated in (a), but make it clear why the behavior is general.\\

\noindent{\bf (c)} What happens to the apocenter radius and the
eccentricity in (a)? Investigate numerically and explain what is
happening.\\

\noindent{\bf (d)} Does the behavior you see change if you apply the
velocity offset at different parts of the orbit (e.g., at apocenter,
or at a random point along the orbit)?\\

\noindent{\bf Problem 3:} More on the virial theorem.\\

\noindent{\bf (a)} In Sec. 4.1 of the lecture notes, we derived the
virial theorem for a set of tracer particles in a point-mass potential
and used it to estimate the Milky Way's mass out to about 40 kpc using
globular clusters. Suppose instead that the potential is that
corresponding to a power-law potential with $\rho(r) \propto
r^{-1}$. Derive the virial estimator for the mass out to a certain
radius $r_c$.\\

\noindent{\bf (b)} Using this estimator, what is the mass of the Milky
Way out to about 40 kpc implied by the globular cluster kinematics?\\

\noindent{\bf (c)} A more realistic model for the Milky Way's mass
distribution is that of a logarithmic potential $\Phi(r) = v_c^2 \,\ln
r$, because this potential has a flat rotation curve. What is the
virial theorem in this case?\\

\noindent{\bf (d)} Using the logarithmic-potential virial theorem,
what is the mass of the Milky Way out to about 40 kpc implied by the
globular cluster kinematics?\\

\end{document}
