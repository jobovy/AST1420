\documentclass[12pt]{article}
\usepackage[letterpaper,margin=1in]{geometry}
\usepackage{hyperref}
\usepackage{amsmath}
\usepackage{amssymb}
\begin{document}
\begin{center}
{\bf \LARGE AST1420 ``Galactic Structure and Dynamics'' Problem Set 2}\\[7pt]
\emph{Due on Oct. 27 at the start of class}\\[7pt]
\end{center}

Most of the exercises in this problem set must be solved on a computer
and the best way to hand in the problem set is as an \texttt{jupyter
  notebook}. Rather than sending me the notebook, you can upload it to
\texttt{GitHub}, which will automatically render the notebook. Rather
than starting a repository for a single notebook, you can upload your
notebook as a \texttt{\href{https://gist.github.com/}{gist}}, which
are version-controlled snippets of code that can optionally be made
private.

If you want to upload your notebook as a gist from the command-line,
you can use the package \href{http://github.com/defunkt/gist}{at this
  http URL} and use it as follows. Log into your \texttt{GitHub}
account:\\

\texttt{gist --login}\\

and then upload your notebook
\texttt{AST1420\_2017\_PS2\_YOURNAME.ipynb} as\\

\texttt{gist -p AST1420\_2017\_PS2\_YOURNAME.ipynb}\\

(the \texttt{-p} option will make the gist private). If you want to
make further changes, you can clone your gist in a separate directory
and use it as you would any other git repository. \emph{Please re-run
  the entire notebook (with \texttt{Cell > Run All}) after re-starting
  the notebook kernel before uploading it}; this will make sure that
the input and output are fully consistent. 

If you are unfamiliar with notebooks, you can also hand in a
traditional write-up, but you also need to send in well-commented code
for how you solved the problems. Thus, notebooks are strongly
preferred :-)\\

\noindent{\bf Problem 1:} (10 points) The total dark-matter halo mass
of Dragonfly 44. In class, we discussed the ultra-diffuse galaxy
Dragonfly 44, which has a total mass of $M(<4.6\,\mathrm{kpc}) =
0.72\pm0.22\times10^{10}\,M_\odot$. Here we will investigate what this
implies for the total dark-matter halo mass of Dragonfly 44.\\

\noindent{\bf (a)} An alternative parameterization of an NFW halo is
through its \emph{virial mass} $M_\mathrm{vir}$ and its concentration
$c$. The virial mass is the mass enclosed within the \emph{virial
  radius} $r_\mathrm{vir}$, which is defined as the radius within
which the mean density of the NFW halo is some multiple $\Delta_h$ of
the mean-matter density $\bar{\rho} = \rho_\mathrm{crit}\,\Omega_m$ in
the Universe (where $\rho_\mathrm{crit}$ is the critical density and
$\Omega_m$ is the matter density parameter). For the purposes of this
exercise, use $H = 70\,\mathrm{km\,s}^{-1}\,\mathrm{Mpc}^{-1}$,
$\Omega_m = 0.3$, and $\Delta_h = 200$. Setting $c=12$, compute the
virial radius (in kpc) for a halo with $M_\mathrm{vir} =
10^{12}\,M_\odot$.\\

\noindent{\bf (b)} Plot the enclosed mass curves for $M_\mathrm{vir} =
10^{10}, 10^{11}, 10^{12}, 10^{13}\,M_\odot$ assuming $c=12$ over the
range $1\,\mathrm{kpc} \leq r \leq 100\,\mathrm{kpc}$.\\

\noindent{\bf (c)} Now plot the same curves, but also add for each
curve the range spanned by reducing the concentration to $c=9$ and
increasing it to $c=16$ (this is roughly the scatter in the
concentration found in numerical simulations; we are ignoring the
minor dependence of concentration on mass here). Also add the observed
enclosed-mass data point for Dragonfly 44. You should end up with
something that looks similar to Figure 5 in van Dokkum et al. (2016;
ApJL, 828, 6). Comment on how strongly the observational point
constrains the virial mass.\\

\noindent{\bf Problem 2:} (10 points) The Kuzmin disk. We discussed
the Kuzmin disk as a simple example of a razor-thin disk potential. We
explore it and orbits in it a little further in this problem.\\

\noindent{\bf (a)} Draw contours of the Kuzmin potential with $a=2$ in
the $(x,z)$ plane \emph{without explicitly evaluating the potential}
(i.e., do not just evaluate the potential and contour this).\\

\noindent{\bf (b)} Prolate spheroidal coordinates are defined by $R =
\Delta\,\sinh u\,\sin v\,, \ z = \Delta\,\cosh u\,\cos v\,,$ where
$\Delta$ is a constant. Show that

\begin{equation}\label{eq:prolate-rz}
  R^2 + (|z|+\Delta)^2  = \Delta^2\,(\cosh u + |\cos v|)^2\,.
\end{equation}

\noindent{\bf (c)} Use Equation $\eqref{eq:prolate-rz}$ to express the
Kuzmin potential as (choose $\Delta$ wisely)

\begin{equation}
  \Phi(u,v) = -\frac{G\,M}{a}\,\frac{\cosh u - |\cos v|}{\sinh^2 u + \sin^2 v}\,.
\end{equation}

\noindent{\bf (d)} Write down the Lagrangian
$\mathcal{L}(u,v,\dot{u},\dot{v})$ and find the relation between the
momenta $(p_u,p_v)$ and $(p_R,p_z) = (\dot{R},\dot{z})$.\\

\noindent{\bf (e)} Write down the Hamiltonian $H(u,v,p_u,p_z)$. What
do you notice about this Hamiltonian? Show that the equation
expressing that the Hamiltonian is equal to the conserved energy,
$H=E$, can be written as

\begin{equation}\label{eq:i3}
  2\Delta^2\,\left[E\,\sinh^2 u + \frac{GM}{a}\,\cosh u\right] -p_u^2-\frac{L_z^2}{\sinh^2 u} = \frac{L_z^2}{\sin^2 v}+p_v^2-2\Delta^2\,\left[E\,\sin^2 v -\frac{GM}{a} |\cos v|\right]\,.
\end{equation}

Because the left-hand side only depends on $u$ and the right-hand side
only depends on $v$, this means that both sides must equal a constant,
which we define to be $2\,\Delta^2\,I_3$. This $I_3$ is a third
integral of the motion in addition to $(E,L_z)$ and the Kuzmin disk
therefore has three explicit integrals of the motion.\\

\noindent{\bf (f)} Setup a Kuzmin potential in \texttt{galpy} with $GM
= 1.8$ and $a = 0.7$. Integrate an orbit with initial conditions
$(R,v_R,v_T,z,v_z,\phi) = (1,0.2,0.9,0,0.1,0)$ in this potential from
$t=0$ to $t=20$, obtaining 10,001 points along the orbit (because of
the force discontinuity at $z=0$, the default orbit integration in
\texttt{galpy} does not conserve energy well enough for this orbit,
use the parameter \texttt{dt=0.000001} in \texttt{Orbit.integrate} to
make sure that a small enough step size is used for the orbit
integration). Plot the following
\begin{itemize}
\item The orbit in $(R,z)$
\item The orbit in $(u,v)$
\item The orbit in $(u,p_u)$
\item The orbit in $(v,p_v)$
\item $I_3$ vs. time using the left-hand side of Equation
  $\eqref{eq:i3}$ and overplot $I_3$ vs. time using the right-hand
  side of the same equation.
\end{itemize}

Comment on what you see in these plots.

\end{document}
