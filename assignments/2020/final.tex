\documentclass[12pt]{article}
\usepackage[letterpaper,margin=1in]{geometry}
\usepackage{hyperref}
\usepackage{url}
\urldef\galpyconfig\url{https://docs.galpy.org/en/latest/installation.html#configuration-file}
\usepackage{amsmath,amssymb}
\usepackage{graphicx}
\begin{document}
\begin{center}
{\bf \LARGE AST1420 ``Galactic Structure and Dynamics'' Final}\\[7pt]
\emph{Due on Dec. 17 at 5pm}\\[7pt]
\end{center}

The full mark for the final includes the oral defense of the exercises
listed here. The breakdown is: 75\% written solutions and 25\,\% oral
defense of solutions. The points given for the different problems
below only relate to the 75\,\% written-solutions part of the full
mark.\\

Some of the exercises in this final must be solved on a computer and
the best way to hand in the final is as an \texttt{ipython
  notebook}. Rather than sending me the notebook, you can upload it to
\texttt{GitHub}, which will automatically render the notebook. Rather
than starting a repository for a single notebook, you can upload your
notebook as a \texttt{\href{https://gist.github.com/}{gist}}, which
are version-controlled snippets of code that can optionally be made
private. If you want to make further changes, you can clone your gist
in a separate directory and use it as you would any other git
repository. \emph{Please re-run the entire notebook (with \texttt{Cell
    > Run All}) after re-starting the notebook kernel before uploading
  it}; this will make sure that the input and output are fully
consistent.

If you are unfamiliar with notebooks, you can also send in a
traditional write-up (in LaTeX), but you also need to send in
well-commented code for how you solved the problems. Thus, notebooks
are strongly preferred :-)\\

\noindent{\bf Problem 1:} (25 points, 5 each) Short questions.\\

\noindent{\bf (a)} In class, we discussed the NFW and Hernquist members of the family of two-power density models. Another commonly used member of this family is the \emph{Jaffe} profile, which can be used to describe the light and mass profile of cuspy elliptical galaxies. The density of the Jaffe profile is
\begin{equation}
  \rho(r) = \frac{\rho_0\,a^2}{r^2\,(1+r/a)^2}\,.
\end{equation}
For this profile, compute the enclosed mass as a function of $r/a$, determine the total mass $M$ in terms of $\rho_0$ and $a$, and compute the gravitational potential and circular velocity as a function of $r/a$, expressing these in terms of $(M,a)$ rather than $(\rho_0,a)$.\\

\noindent{\bf (b)} Show that the velocity dispersion for an isotropic spherical system can be calculated as
\begin{equation}
\sigma^2 = {1 \over \nu(r)}\,\int_{\Phi(r)}^0\,\mathrm{d} \Phi'\,\nu(\Phi')\,,
\end{equation}
where $\Phi$ is the gravitational potential and $\nu$ is the
density.\\

\noindent{\bf (c)} New observations of the ultra-diffuse galaxy
Dragonfly 44 show that it has a line-of-sight velocity dispersion of
$33\pm3\,\mathrm{km\,s}^{-1}$ at its 3D half-light radius of
$4.7\pm0.2\,\mathrm{kpc}$ and the velocity dispersion is approximately
constant with radius. Using this information, determine the total
dynamical mass of Dragonfly 44 within its half-light radius.\\

\begin{figure}[htp]
\includegraphics[width=\textwidth]{siegel.png}
\caption{Observed distribution of [Eu/Fe] vs. [Fe/H] for solar
  neighborhood stars.}\label{fig:siegel}
\end{figure}

\noindent{\bf (d)} The origin of r process elements has long been
mysterious, but neutron-star (NS) mergers were long thought to play a
potential role and observations of the light curve of the double NS
merger GW170817 showed the presence of a significant amount of
r-process material in the merger ejecta. The amount of material
inferred combined with the rough estimate of the occurence rate
obtained from a single NS merger was consistent with \emph{all} r
process elements in the Universe being created in NS mergers.

To get to a NS merger, a binary system must first evolve to a binary
NS system that then slowly spirals in through gravitational wave
radiation unil the NSs merge. This is similar to the favored
double-degenerate scenario for type Ia supernovae, where a binary must
evolve to a double white-dwarf binary which then slowly spirals in
until the white dwarfs merge. The typical time scale for the latter
process is a Gyr and we expect the time scale for the double NS merger
therefore to be the same.

Given that iron in the solar neighborhood for stars with metallicities
like the Sun comes about 50\% from type Ia supernovae and about 50\%
from type II supernovae, while in the double NS merger scenario 100\%
of r process elements like Eu come from a process with similar time
dependence as type Ia supernovae, describe what the expected location
of stars looks like in the plane made up of [Eu/Fe] and [Fe/H] (from
low metallicity $[\mathrm{Fe/H}] \approx -2$ to $\approx 0$; similar
to the [O/Fe] vs. [Fe/H] plane that we discussed in class).

Fig. \ref{fig:siegel} shows the observed distribution of [Eu/Fe]
vs. [Fe/H] in the solar neighborhood. This looks similar to that of
[Mg/Fe] vs. [Fe/H]. What can you conclude from this regarding the
origin of the r process and the NS merger contribution?\\

\noindent{\bf (e)} In Section 16.2, we saw that isotropic rotation
cannot support the oblate structure of most elliptical galaxies. Thus,
anisotropy in the velocity-dispersion tensor is necessary to support
the shape of elliptical galaxies. Assuming that the dispersion tensor
$\Pi_{zz} = (1-\delta)\,\Pi_{xx}$, with $\delta$ the global
anisotropy parameter, derive the equivalent of relation (16.68) for
$\delta \neq 0$ and plot the resulting relations for $\delta \in
\{0.0,0.1,0.2,0.3,0.4,0.5\}$. Also draw curves for different ratios
$\varepsilon_{\mathrm{int}}/\varepsilon_{\mathrm{obs}}$ as we did in
the text, making use of the fact that the equivalent to Equation
(16.73) for $\delta \neq 0$ modifies each occurrence of
$W_{xx}/W_{zz}$ in that equation in the same way. By comparing to the
data on $v/\sigma$ versus ellipticity from van de Sande et al. (2017)
shown in the text, what values of $\delta$ are required to explain the
oblateness of typical elliptical galaxies?\\

\noindent{\bf Problem 2:} (25 points, 5 each) The epicycle
approximation, Bertrand's theorem, and the mass distribution in the
center of the Milky Way.\\

A useful approximation of the orbits in galaxies,
especially those of disk stars in disk galaxies, is the \emph{epicycle
  approximation}. In this approximation, one approximates the
gravitational effective potential by Taylor expanding it to second order
\begin{equation}
    \Phi_\mathrm{eff}(R,z;L_z) \approx \Phi_\mathrm{eff}(R_g,0;L_z)
            + \frac{1}{2}\left(\frac{\partial^2 \Phi_\mathrm{eff}}{\partial R^2}\right)\Bigg|_{(R_g,0)}\,(R-R_g)^2
            +\frac{1}{2}\left(\frac{\partial^2 \Phi_\mathrm{eff}}{\partial z^2}\right)\Bigg|_{(R_g,0)}\,z^2\,,
\end{equation}
around the guiding-center radius of a star. This leads to equations of
motion that are
\begin{align}
   \ddot{R} = \ddot{R}-\ddot{R_g} &= -\left(\frac{\partial^2 \Phi_\mathrm{eff}}{\partial R^2}\right)\Bigg|_{(R_g,0)}\,(R-R_g)\,,\\
   \ddot{z} & =  -\left(\frac{\partial^2 \Phi_\mathrm{eff}}{\partial z^2}\right)\Bigg|_{(R_g,0)}\,z\,,
\end{align}
because $\ddot{R}_g = 0$. These are the equations of a decoupled
harmonic oscillator in $(R-R_g,z)$, with frequencies. The first of
these, $\kappa$, is known as the epicycle frequency or as the radial
frequency, while the second is known as the vertical frequency. A
third important frequency is the circular frequency $\Omega(R_g)$,
which is the azimuthal frequency of the circular orbit at $R_g$ and is
therefore $\Omega(R_g)=v_c(R_g)/R_g=L_z/R^2_g$. As shown in Section
10.3 in the notes, the equations of motion can be solved fully
analytically in terms of these frequencies and the initial conditions
and the resulting motion in $(R,\phi)$ is that of motion along an
ellipse whose center is the guiding center, itself orbiting on a
circular around the center. The vertical motion is a simple harmonic
oscillation decoupled from the planar motion.\\

\noindent{\bf (a)} For a flat rotation curve $v_c(R) = $ constant,
compute $\kappa/\Omega$.\\

\noindent{\bf (b)} Betrand's theorem states that the only mass
distributions for which all orbits close are the (a) point-mass and
(b) homogeneous density sphere. Let's investigate this here and see
what it implies about the mass distribution in the Galactic center.

If all orbits in a mass distribution close, then in particular orbits
that are close to a circular orbit must close. Close-to-circular
orbits are described by the epicycle approximation, so we can use this
approximation to see whether orbits close. Using the epicycle
approximation for spherical mass distributions---the same as that
discussed for disks above, but without the vertical dependence of the
potential---demonstrate that the only gravitational potentials for
which close-to-circular orbits close have $\phi(r) \propto
r^{\beta^2-2}$ where $\beta$ is a rational number (you can start from
the assumption that the potential has a power-law form, because over a
small range of radii, all potentials can be approximated as such).\\

\noindent{\bf (c)} The form $\phi(r) \propto r^{\beta^2-2}$ with
$\beta$ a rational number includes potentials where $\beta=5/3$ or
$\beta=4/3$. Through explicit orbit integration using \texttt{galpy},
investigate whether all orbits close in potentials with $\beta =
[1,16/15,5/4,4/3,3/2,11/6,23/12,2]$. What can you conclude?\\

\noindent{\bf (d)} Now numerically show that what you claim in (c) is
correct by explicitly calculating the radial and azimuthal periods of
well-chosen orbits in each of these potentials.\\

\noindent{\bf (e)} In the Galactic center, we can observe (partial)
orbits of the so-called ``S stars''. In particular, the orbit of the
star S2 (or S0-2 depending on who you ask) has been observed through a
full azimuthal period and its orbit closes to within the
uncertainties. While there are closed orbits in many potentials, the
fact that the one orbit that we observe closes (and which is not
circular) is good evidence that \emph{all} orbits close in the mass
distribution that S2 is orbiting in. From Bertrand's theorem we know
that this means that the mass distribution is dominated either by a
massive point-like object or that it is homogeneous. These are quite
different mass distributions! Discuss what kinds of observations of
the S stars could distinguish between these two possibilities.\\

\end{document}
