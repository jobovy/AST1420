\documentclass[12pt]{article}
\usepackage[letterpaper,margin=1in]{geometry}
\usepackage{hyperref}
\usepackage{amsmath,amssymb}
\begin{document}
\begin{center}
{\bf \LARGE AST1420 ``Galactic Structure and Dynamics'' Problem Set 1}\\[7pt]
\emph{Due on Oct. 15 at the start of class}\\[7pt]
\end{center}

Most of the exercises in this problem set must be solved on a computer
and the best way to hand in the problem set is as an \texttt{ipython
  notebook}. Rather than sending me the notebook, you can upload it to
\texttt{GitHub}, which will automatically render the notebook. Rather
than starting a repository for a single notebook, you can upload your
notebook as a \texttt{\href{https://gist.github.com/}{gist}}, which
are version-controlled snippets of code that can optionally be made
private. If you want to make further changes, you can clone your gist
in a separate directory and use it as you would any other git
repository. \emph{Please re-run the entire notebook (with \texttt{Cell
    > Run All}) after re-starting the notebook kernel before uploading
  it}; this will make sure that the input and output are fully
consistent.

If you are unfamiliar with notebooks, you can also send in a
traditional write-up (in LaTeX), but you also need to send in
well-commented code for how you solved the problems. Thus, notebooks
are strongly preferred :-)\\

\noindent{\bf Problem 1:} The virial mass of the NFW profile. The
virial mass of a dark matter halo is arguably its most fundamental
parameter, because the tight correlation between correlation and mass
found in numerical simulations of dark-matter halo formation means
that dark-matter halos in nature form an essentially one-dimensional
sequence of mass. However, the virial mass depends on how one chooses
the overdensity $\Delta_v$ that defines the virial radius. A standard
value for this is $\Delta_v = 200$, but as we will see when we discuss
the formation of dark-matter halos in more detail, $\Delta_v$ should
depend on the cosmological parameters and the redshift of the halo's
formation. When doing this, $\Delta_v \approx 200$ at high redshift
($z \gtrsim 2$) in our Universe, but at the present day a value of
$\Delta_v \approx 100$ is more correct. For a quantity as fundamental
as the virial mass, many discussions of it in papers and elsewhere are
surprisingly vague on the overdensity used to define it! Let's see how
much of an issue this is.\\

(a) Using the equations given for the NFW profile and using values
appropriate for the Milky Way's dark-matter halo ($\rho_0 =
0.0035\,M_\odot\,\mathrm{pc}^{-3}$ and $a = 16\,\mathrm{kpc}$),
compute the virial radius and virial mass as a function of $\Delta_v$
and plot them. Discuss how the virial radius and virial mass depend on
$\Delta_v$. Use $H_0 = 70\,\mathrm{km\,s}^{-1}\,\mathrm{Mpc}^{-1}$.\\

(b) What about the NFW density profile causes the behavior that you
see?\\

\noindent{\bf Problem 2:} In his colloquium a few weeks ago, Scott
Tremaine discussed scattering of comets by the planets in the solar
system as the comets pass through the inner solar system. He mentioned
that scattering of the comets tends to preserve their pericentric
distances. Let's understand why that is using what we know about
orbits in spherical potentials!\\

\noindent{\bf (a)} Consider an orbit in a spherical isochrone
potential with $b=1$ (pick an orbit that explores $r \approx 1$ and is
not too close to circular). Using orbit integration in \texttt{galpy},
add an instantaneous velocity offset when the orbit is at its
pericenter radius. Investigate what happens to the pericenter radius
of the resulting orbit. Is it larger or smaller than the original
pericenter radius? It is useful to consider the special cases where
(i) you only change the radial velocity and (ii) you only change the
tangential velocity (or equivalently the angular momentum). Does the
answer change if you consider different orbits?\\

\noindent{\bf (b)} Argue why the behavior you saw in part (a) is true
for \emph{any} orbit in \emph{any} spherical potential. (Hint:
consider the special cases and what happens to the effective
potential). You can illustrate your argument with the orbit(s) that
you investigated in (a), but make it clear why the behavior is general.\\

\noindent{\bf (c)} What happens to the apocenter radius and the
eccentricity in (a)? Investigate numerically and explain what is
happening.\\

\noindent{\bf (d)} Now consider the orbit of a comet that originates
from the Oort cloud, at 20,000 AU and has an orbit that brings it to 2
AU. The perturbations to the orbit near its pericenter from the
planets lead to changes in the energy that are equivalent to changing
the inverse semi-major axis by $10^{-4}\,\mathrm{AU}^{-1}$. By
considering a few different ways of distributing this energy change
into radial and tangential velocity kicks, determine how the
pericenter and apocenter distances of this comet change. Do you see
what Scott claimed? Discuss.\\

\noindent{\bf Problem 3:} The cored isothermal sphere and
self-interacting dark matter models.\\

\noindent{\bf (a)} Equation (6.71) for the density of an isothermal
sphere has non-singular solutions that can be found by specifying the
boundary condition for a core: $\rho(0) = \rho_0$ and $\mathrm{d} \rho
/ \mathrm{d} r = 0$ at $r=0$. Demonstrate by writing Equation (6.71)
in terms of $y=\ln \tilde{\rho}/\rho_0$ and $x = r/r_0$ where $r_0^2 =
9\sigma^2/[4\pi G\rho_0]$ that cored solutions have the form $\rho(r)
= \rho_0\,f(r/r_0)$ and give the equation that determines $f(x)$.\\

\noindent{\bf (b)} Write a function that computes $f(x)$ and use it to
plot $\rho/\rho_0$ as a function of $r/r_0$ for the cored isothermal
sphere. Compare what you see to the singular isothermal sphere.\\

\noindent{\bf (c)} An often preferred model for the dark matter
density profile is the NFW profile. Using an NFW profile with
concentration 11.5 and a virial mass of $7\times 10^{11}\,M_\odot$
(for $\Delta_v = 200$; this is like the Milky Way's dark matter halo),
numerically compute the radial velocity dispersion profile in
$\mathrm{km\,s}^{-1}$ for $\beta = 0$ and $\beta = 0.5$ (implement the
integrals yourself, don't just use \texttt{galpy.df.jeans}). Plot your
solution on a logarithmic grid from $r = 1\,\mathrm{kpc}$ to
$r=300\,\mathrm{kpc}$.\\

\noindent{\bf (d)} The cored isothermal sphere describes the inner
regions of dark matter halos in models where dark matter particles
interact strongly enough that they scatter off of each other and
thermalize in regions of high enough dark-matter density (through
interactions that are analogous to non-gravitational interactions
between baryons). This thermalization homogenizes the velocity
dispersion and this means that in these models, the outer dark matter
profile is given by the standard NFW form, while the inner profile is
that of the cored isothermal sphere. The boundary between these two
regimes is at the radius where a dark matter particle is expected to
scatter once. The scattering rate per unit time per particle is given by

\begin{equation}
  \Gamma(r) = {\sigma \over m}\,{4 \over \sqrt{\pi}}\,\sigma_r(r)\,\rho(r)\,,
\end{equation}

where $\sigma/m$ is the self-interacting dark matter cross section per
unit mass (note that this is \emph{not} the same $\sigma$ as that of
the isothermal sphere, but $\sigma/m$ is the standard notation for
this cross section), $\sigma_r(r)$ is the radial velocity dispersion,
and $\rho(r)$ the density profile. For $\sigma/m =
1\,\mathrm{cm^2\,g}^{-1}$, a halo age of 10 Gyr, and the NFW halo and
radial-velocity dispersion profiles from (c), determine the radius
$r_1$ at which a particle in the NFW halo is expected to scatter once,
both for $\beta=0$ and $\beta=0.5$.\\\

\noindent{\bf (e)} Given the cored-isothermal sphere profile that you
found in (b), find the cored-isothermal profile (that is, the
parameters $\rho_0$ and $r_0$) such that the cored-isothermal
profile's density and enclosed mass matches that of the NFW profile
from (c) and (d) at $r_1$ (for the $\beta=0$ case for the NFW's
velocity dispersion). Plot the entire density profile of the
cored-isothermal profile out to $r_1$ and the NFW profile outside of
that from 1 kpc to 300 kpc. This is a simple model for the Milky Way's
dark matter halo if dark matter has strong self interactions!

\end{document}
