\documentclass[12pt]{article}
\usepackage[letterpaper,margin=1in]{geometry}
\usepackage{hyperref}
\usepackage{amsmath}
\usepackage{amssymb}
\begin{document}
\begin{center}
{\bf \LARGE AST1420 ``Galactic Structure and Dynamics'' Final}\\[7pt]
\emph{Due at 5pm on Dec. 13}\\[7pt]
\end{center}

The full mark for the final includes the oral defense of the exercises
listed here as well as your oral answers to the selected questions
from the Astronomy \& Astrophysics General Qualifying Exam, listed at
the end of this document. The breakdown is: 50\% written solutions,
25\,\% oral defense of solutions, 25\,\% Qualifying Exam
questions. The points given for the different problems below only
relate to the 50\,\% written-solutions part of the full mark.\\

Some of the exercises in this final must be solved on a computer and
the best way to hand in the final is as an \texttt{jupyter
  notebook}. Rather than sending me the notebook, you can upload it to
\texttt{GitHub}, which will automatically render the notebook. Rather
than starting a repository for a single notebook, you can upload your
notebook as a \texttt{\href{https://gist.github.com/}{gist}}, which
are version-controlled snippets of code that can optionally be made
private.

If you want to upload your notebook as a gist from the command-line,
you can use the package \href{http://github.com/defunkt/gist}{at this
  http URL} and use it as follows. Log into your \texttt{GitHub}
account:\\

\texttt{gist --login}\\

and then upload your notebook
\texttt{AST1420\_2017\_FINAL\_YOURNAME.ipynb} as\\

\texttt{gist -p AST1420\_2017\_FINAL\_YOURNAME.ipynb}\\

(the \texttt{-p} option will make the gist private). If you want to
make further changes, you can clone your gist in a separate directory
and use it as you would any other git repository. \emph{Please re-run
  the entire notebook (with \texttt{Cell > Run All}) after re-starting
  the notebook kernel before uploading it}; this will make sure that
the input and output are fully consistent. 

If you are unfamiliar with notebooks, you can also hand in a
traditional write-up, but you also need to send in well-commented code
for how you solved the problems. Thus, notebooks are strongly
preferred :-)\\

\noindent{\bf Problem 1:} (25 points, 5 each) Short questions.\\

\noindent{\bf (a)} Consider the following potential
\begin{equation}
  \Phi(r) = -4\pi G\,\rho_0\,a^2\,\ln\left[1+\frac{a}{r}\right]\,,
\end{equation}
where $a$ and $\rho_0$ are constants. Without directly computing the
full density profile, determine (i) whether or not the total mass is
finite and if it is finite, what the mass is; (ii) the large $r$
power-law behavior limit of the density.\\

\noindent{\bf (b)} Give an initial phase-space position in cylindrical
coordinates for an orbit with eccentricity $0.50$ that reaches a
maximum height of $20\,\mathrm{kpc}$ above the $z=0$ plane in an NFW
potential with total mass $M = 10^{12}\,M_\odot$ and scale radius $a =
16\,\mathrm{kpc}$. Verify your answer by setting up this problem in
\texttt{galpy}, integrating the orbit, and computing its eccentricity
and maximum height using the orbit methods \texttt{e} and
\texttt{zmax}.\\

\noindent{\bf (c)} (BT08 3.15) Prove that at any point in an
axisymmetric system at which the local density is negligible, the
epicycle, vertical, and circular frequencies $\kappa$, $\nu$, and
$\Omega$ are related by $\kappa^2+\nu^2 = 2\Omega^2$.\\

\noindent{\bf (d)} In certain dark-matter models, dark matter
particles feel an additional long-range force mediated through a
scalar particle. Such an interaction gives rise to a Yukawa
potential $V_y(r)$
\begin{equation}
  V_y(r) = -\frac{g^2}{4\pi\,r}\,e^{-m_\phi\,r}\,,
\end{equation}
where $m_\phi$ is the mass of the scalar mediator and $g$ is a
coupling constant (we are using units in which $\hbar = c = 1$
here). At scales $r \gg m_\phi^{-1}$, such a force acts as an
additional gravitational force between dark-matter particles and can
therefore be modeled as a simple change in the gravitational constant
for such particles: $\tilde{G} = G\,(1+\beta^2)$, where $\beta =
g\,m_P/\sqrt{4\pi}m_{\mathrm{DM}}$ (with $m_P$ the Planck
mass). Discuss how this additional interaction affects the following
(that is, how does the inferred mass relate to the true mass):

\begin{itemize}
\item The matter distribution in disk galaxies inferred from their
  rotation curves.
\item The mass of the Milky Way inferred from the velocities of halo globular clusters.
\item The mass of the Milky Way inferred by assuming that a distant satellite (e.g., Leo I) moves at the escape velocity.
\item The mass of the Local Group determined using the timing
  argument.
\end{itemize}

\noindent{\bf (e)} In the Milky Way, the ratio of the
epicycle-to-angular frequency $\kappa/\Omega \approx 4/3$. Assuming
that the rotation curve depends on radius as $v_c \propto r^\alpha$
and that $\kappa/\Omega$ is constant at the above value, what is the
outer Lindblad radius $R_\mathrm{OLR}$ of an $m=2$ bar with $\Omega_b
= 1.85\,\Omega_0$ relative to the Solar radius $R_0$ (where the
angular velocity is $\Omega_0$)?\\

\noindent{\bf Problem 2:} (15 points; 3 each) Disk galaxies as
collisionless sytems? One of the first topics that we discussed and
something we relied on throughout the course is that galaxies are
\emph{collisionless systems}, that is, their two-body relaxation time
$t_\mathrm{relax}$ is orders of magnitude larger than the current age
of the Universe. We derived this result for a simple, uniform density
spherical system and argued that there are enough orders of magnitude
to spare in this argument, such that these simplifying assumptions do
not affect the result. Let's look into that a little more closely
now.\\

\noindent{\bf (a)} We derived the relaxation time $t_\mathrm{relax}$
for a spherical system, but many galaxies are \emph{disk galaxies}
that are far from spherical. Derive the relaxation time in the
two-dimensional limit, that is, for a uniform density, razor-thin
galactic disk with size $R$ and typical velocity $v$. What is the
ratio of $t_\mathrm{relax}$ to $t_\mathrm{cross}$?\\

\noindent{\bf (b)} Explain the result in (a) physically: why is the
situation in case (a) fundamentally different from the spherical case?
(Hint: Newton's theorems and the potential of razor-thin disks).\\

\noindent{\bf (c)} The result in (a) might surprise you. But the
situation gets even more interesting. Galactic disks are not just
close to being two dimensional, they are also cold in the sense that
the random velocity $\sigma$ of stars is a small fraction of the
orbital velocity $v$; let's call $\alpha = \sqrt{2}\sigma/v$ the
typical relative velocity between two stars. How does the relaxation
time that you derived in (a) change for a cold disk, assuming a
constant $\alpha$?\\

\noindent{\bf (d)} In reality, galactic disks are not razor-thin, but
have some finite thickness $z_d$ in addition to their two-dimensional
size $R_d$. Derive the relaxation time for a cold ($\alpha < 1$)
finite-thickness disk. What is this relaxation time relative to that
for a spherical system? Remember: we are only interested in an
order-of-magnitude estimate of the relaxation time, so do not attempt
a complicated derivation, but rather make reasonable simplifying
assumptions.\\

\noindent{\bf (e)} Using typical parameters for the Milky Way disk, is
the disk collisionless? How many particles $N$ are necessary to
simulate its dynamical evolution as a collisionless system?\\

\noindent{\bf Problem 3:} (10 points) Conservation laws and numerical
integration. Consider an $N$-body system under unsoftened, Newtonian
gravity. Write a simple $N$-body integrator that uses i) direct
summation to obtain the forces and ii) the modified Euler, leapfrog,
or fourth-order Runge-Kutta method for orbit integration of the $N$
bodies. Investigate the conservation of total angular momentum $\sum_i
m_i\,\vec{x}_i\times\vec{v}_i$ when using a constant (small) time
step. (Note: you are free to choose $N$, but do not make it \emph{too}
small). You should evolve your system for about 100 dynamical
times. Explain what you see in terms of the theory behind the
different integrators.

\newpage

\section*{Selected questions from the General Qualifying Exam:}

After having this course, you should be able to answer the following
questions from the General Qualifying Exam:

\begin{itemize}
\item What is the total mass (in both dark matter and in stars) of the
  Milky Way galaxy? How does this compare to M31 and to the LMC? How
  is this mass determined?
\item Define and describe globular clusters. Where are they located?
  What are their typical ages, and how is this determined?
\item Describe a dynamical method used in the determination of the
  mass of a galaxy cluster. (Note: full question asks for three
  methods, not solely dynamical).
\item What are galaxy clusters? What are their basic properties (e.g.,
  mass, size). [We'll skip this part: List and explain three ways they
    can be detected.]
\item Describe the orbits of stars in a galactic disk and in a
  galactic spheroid.
\item Galactic stars are described as a collisionless system. Why?
\item What is the difference between a globular cluster and a dwarf
  spheroidal galaxy?
\item The stars in the solar neighborhood, roughly the 300 pc around
  us, have a range of ages, metallicities, and orbital properties. How
  are those properties related?
\item What is dynamical friction? Explain how this operates in the
  merger of a small galaxy into a large one.
\item Sketch the rotation curve for a typical spiral galaxy. Show that
  a flat rotation curve implies the existence of a dark matter halo
  with a density profile that drops off as $1/r^2$.
\item Characterize the stellar populations in the following regions:
  i) the Galactic bulge, ii) the Galactic disk, outside of star
  clusters, iii) open star clusters, iv) globular clusters, v) a
  typical elliptical galaxy.
\item What is the G-dwarf problem in the solar neighborhood?
\end{itemize}

\end{document}
